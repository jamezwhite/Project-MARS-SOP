%%%%%%%%%%%%%%%%%%%%%%%%%%%%%%%%% Preamble %%%%%%%%%%%%%%%%%%%%%%%%%%%%%%%%%
\documentclass[]{report}

\usepackage{titling}               % Adds control for typestting of the \maketitle command
\usepackage{newtxtext}
\usepackage{fancyhdr}              % Gives Control over headers and footers
\usepackage{acro}                  % Adds Acronym Glossing and Appendix
\usepackage[section=section]{glossaries}            % Adds Term Glossing and Appendix
\usepackage{blindtext}             % Creates Lorem Ipsum when asked for
\usepackage{titlesec}              % Allows for Alternative Section Titles
\usepackage{graphicx}              % Allows for image use
\usepackage[titletoc]{appendix}    % Gives extra control of appendices
\usepackage[
  letterpaper,
  margin=1in,          % set all margins; tweak to taste (e.g., 0.8in)
  includeheadfoot,     % count header & footer space inside the margins
  headheight=14pt,     % matches your fancyhdr header content height
  headsep=10pt,        % space between header and text
  footskip=24pt        % space from bottom of text to baseline of footer
]{geometry}                        %Flexible and Complete Interface to Document Dimensions
\usepackage[hidelinks]{hyperref}
\usepackage{cleveref}              %Allows for intelligent crossreferencing
\usepackage{pdfcomment}            % Allows for Tooltips in PDF
\usepackage{enumitem}              % for tidy list spacing (optional)


\renewcommand{\familydefault}{\sfdefault} % Use sans-serif font

%%%%%%%%%%%%%%%%%%%%%%%%%%%%%%%%% Header/Footer Formatting %%%%%%%%%%%%%%%%%%%%%%%%%%%%%%%%%

\pagestyle{fancy} % use fancyhdr package to customize headers/footers
\fancyhf{} % clear all header and footer fields

% Expose the date you set for the title page so we can reuse it in the footer
\makeatletter
\newcommand{\frontpagedate}{\@date} % this will mirror whatever you put in \date{...}
\makeatother

% Put ONLY the chapter title (not "Chapter 1") in the running header mark
\renewcommand{\chaptermark}[1]{\markboth{#1}{}}

% --- Header: top-left shows chapter title on non-opening pages
\fancyhead[L]{\nouppercase{\leftmark}}

% --- Footer: left = fixed text + front-page date, right = page number
\fancyfoot[L]{Project MARS SOP, \thedate}
\fancyfoot[R]{\thepage}

% Optional rules (lines). Header rule on; footer rule off.
\renewcommand{\headrulewidth}{0.4pt}
\renewcommand{\footrulewidth}{0pt}

% Make chapter opening pages use a blank header but keep the same footer
% (report class sets chapter openings to \thispagestyle{plain} by default)
\fancypagestyle{plain}{%
  \fancyhf{}%
  \fancyfoot[L]{Project MARS SOP, \thedate}%
  \fancyfoot[R]{\thepage}%
  \renewcommand{\headrulewidth}{0pt}%
}

% Prevent the classic "Package Fancyhdr Warning: \headheight is too small"
\setlength{\headheight}{14pt} % bump if you add taller content

%%%%%%%%%%%%%%%%%%%%%%%%%%%%%%%%% Acronym Setup and Tooltips %%%%%%%%%%%%%%%%%%%%%%%%%%%
% Load all acronyms from source file
\DeclareAcronym{acm}{
	short = {ACM} ,
	long = {Aircrew Member}
}

\DeclareAcronym{agl}{
	short = {AGL} ,
	long = {Above Ground Level}
}

\DeclareAcronym{ao}{
	short = {AO} ,
	long = {Autonomous Operator}
}

\DeclareAcronym{arng}{
	short = {ARNG} ,
	long = {Army National Guard}
}

\DeclareAcronym{atc}{
	short = {ATC} ,
	long = {Air Traffic Control}
}

\DeclareAcronym{atp}{
	short = {ATP} ,
	long = {Aircrew Training Program}
}

\DeclareAcronym{awr}{
	short = {AWR} ,
	long = {Air Worthiness Release}
}

\DeclareAcronym{buq}{
	short = {BUQ } ,
	long = {Basic UAS Qualification}
}

\DeclareAcronym{bvlos}{
	short = {BVLOS} ,
	long = {Beyond Visual Line-of-Sight}
}

\DeclareAcronym{c2}{
	short = {C2} ,
	long = {Command and Control}
}

\DeclareAcronym{coa}{
	short = {COA} ,
	long = {Certificate of Authorization}
}

\DeclareAcronym{dha}{
	short = {DHA} ,
	long = {Defense Health Agency}
}

\DeclareAcronym{dod}{
	short = {DoD} ,
	long = {Department of Defense}
}

\DeclareAcronym{dte}{
	short = {DT\&E} ,
	long = {Developmental Test \& Evaluation Flights}
}

\DeclareAcronym{eo}{
	short = {EO} ,
	long = {Expert Operator}
}

\DeclareAcronym{faa}{
	short = {FAA} ,
	long = {Federal Aviation Administration}
}

\DeclareAcronym{frago}{
	short = {FRAGO} ,
	long = {fragmentary order}
}

\DeclareAcronym{gcs}{
	short = {GCS} ,
	long = {Ground Control Station}
}

\DeclareAcronym{gps}{
	short = {GPS} ,
	long = {Global Positioning System}
}

\DeclareAcronym{isr}{
	short = {ISR} ,
	long = {Intelligence, Surveillance, and Reconnaisance}
}

\DeclareAcronym{kach}{
	short = {KACH} ,
	long = {Keller Army Community Hospital}
}

\DeclareAcronym{mars}{
	short = {MARS} ,
	long = {Medical Autonomous Resupply System}
}

\DeclareAcronym{mchc}{
	short = {MCHC} ,
	long = {Mologne Cadet Health Clinic}
}

\DeclareAcronym{mfr}{
	short = {MFR} ,
	long = {Memorandum for Record}
}

\DeclareAcronym{mgtow}{
	short = {MGTOW} ,
	long = {Max Gross Takeoff Weight}
}

\DeclareAcronym{mou}{
	short = {MoU} ,
	long = {Memorandum of Understanding}
}

\DeclareAcronym{msl}{
	short = {MSL} ,
	long = {Mean Sea Level}
}

\DeclareAcronym{mt}{
	short = {MT} ,
	long = {Master Trainer}
}

\DeclareAcronym{nas}{
	short = {NAS} ,
	long = {National Airpace System}
}

\DeclareAcronym{ncoic}{
	short = {NCOIC} ,
	long = {Non-Commissioned Officer in Charge}
}

\DeclareAcronym{notam}{
	short = {NOTAM} ,
	long = {Notice to Airmen}
}

\DeclareAcronym{nsufr}{
	short = {NSUFR} ,
	long = {National Security UAS Flight Restriction}
}

\DeclareAcronym{oic}{
	short = {OIC} ,
	long = {Officer in Charge}
}

\DeclareAcronym{oop}{
	short = {OOP} ,
	long = {Operations Over People}
}

\DeclareAcronym{opord}{
	short = {OPORD} ,
	long = {operation order}
}

\DeclareAcronym{qp}{
	short = {QP} ,
	long = {Qualified Pilot}
}

\DeclareAcronym{rpic}{
	short = {RPIC} ,
	long = {Remote Pilot in Charge}
}

\DeclareAcronym{rth}{
	short = {RTH} ,
	long = {Return-to-Home}
}

\DeclareAcronym{rwc}{
    short = {RWC} ,
    long = {Remain Well Clear}
}

\DeclareAcronym{sop}{
	short = {SOP} ,
	long = {Standard Operating Procedure}
}

\DeclareAcronym{srd}{
	short = {SRD} ,
	long = {Systems Readiness Directorate}
}

\DeclareAcronym{suas}{
	short = {sUAS} ,
	long = {Small Unmanned Aircraft System}
}

\DeclareAcronym{tfb}{
	short = {TFB} ,
	long = {Test-Flight Box}
}

\DeclareAcronym{tfr}{
	short = {TFR} ,
	long = {Temporary Flight Restriction}
}

\DeclareAcronym{ua}{
	short = {UA} ,
	long = {Unmanned Aircraft}
}

\DeclareAcronym{uac}{
	short = {UAC} ,
	long = {Unmanned Aircraft Crewmember}
}

\DeclareAcronym{uas}{
	short = {UAS} ,
	long = {Unmanned Aircraft System}
}

\DeclareAcronym{uav}{
	short = {UAV} ,
	long = {Unmanned Aerial Vehicle}
}

\DeclareAcronym{usaasa}{
	short = {USAASA} ,
	long = {US Army Aeronautical Services Agency}
}

\DeclareAcronym{usar}{
	short = {USAR} ,
	long = {United States Army Reserve}
}

\DeclareAcronym{usma}{
	short = {USMA} ,
	long = {United States Military Academy}
}

\DeclareAcronym{usmaps}{
	short = {USMAPS} ,
	long = {United States Military Academy Prepatory School}
}

\DeclareAcronym{vlos}{
	short = {VLOS} ,
	long = {Visual Line of Sight}
}

\DeclareAcronym{vo}{
	short = {VO} ,
	long = {Visual Observer}
}

  

% Redefine commands after everything is loaded to turn on tooltips
\AtBeginDocument{%
    \let\oldac\ac
    \renewcommand{\ac}[1]{%
        \pdftooltip{\oldac{#1}}{\acl{#1}}%
    }%
}

%%%%%%%%%%%%%%%%%%%%%%%%%%%%%%%%% Term Setup and Tooltips %%%%%%%%%%%%%%%%%%%%%%%%%%%
% Load all acronyms from source file
\makenoidxglossaries
\newglossaryentry{corridor}{
	name = {air corridor} ,
	description = {A restricted air route of travel specified for use by friendly aircraft and established for the purpose of preventing friendly aircraft from being fired on by friendly forces.
	\glspar\hfill\textit{Source: JP 3-52, DoD, 2014}}}

\newglossaryentry{aircraft}{
	name = {aircraft} ,
	description = {A device that is used, or intended to be used, for flight.
	\glspar\hfill\textit{Source: Pilot's Handbook of Aeronautical Knowledge, FAA, November 2023}}}

\newglossaryentry{airframe}{
	name = {airframe} ,
	description = {The fuselage, booms, nacelles, cowlings, fairings, airfoil surfaces (including rotors but excluding propellers and rotating airfoils of engines), and landing gear of an aircraft and their accessories and controls.
	\glspar\hfill\textit{Source: 14 CFR Subsection 1.1, US Congress, September 2025}}}

\newglossaryentry{automated}{
	name = {automated} ,
	description = {Operated by machines or computers without direct human control.
	\glspar\hfill\textit{Source: Dictionary, Merriam-Webster, 2024}}}

\newglossaryentry{adsb}{
	name = {Automatic Dependent Surveillance - Broadcast (ADS-B)} ,
	description = {A function of an aircraft or vehicle that preiodically broadcasts its state vector (i.e., horizontal and vertical position, horizontal and vertical velocity) and other information.
	\glspar\hfill\textit{Source: Pilot's Handbook of Aeronautical Knowledge, FAA, November 2023}}}

\newglossaryentry{autonomousflight}{
	name = {autonomous flight} ,
	description = {The operation of a /ac{uas} where a complex function makes decisions and performs actions without direct human input, enabled by a run-time assurance (RTA) architecture that safely bounds the flight behavior.
	\glspar\hfill\textit{Source: ASTM F3269-17, ASTM International, 2017}}}

\newglossaryentry{bvlos}{
	name = {Beyond Visual Line of Sight (BVLOS)} ,
	description = {The operation of a UAS beyond the visual capability of the flight crew members (i.e., remote pilot in command [RPIC], the person manipulating the controls, and visual observer [VO]), if used to see the aircraft with vision unaided by any device other than corrective lenses, spectacles, and contact lenses.
	\glspar\hfill\textit{Source: Pilot/Controller Glossary, FAA, August 2025}}}

\newglossaryentry{coa}{
	name = {Certificate or Waiver of Authorization (CoA)} ,
	description = {An FAA grant of approval for a specific flight operation or airspace authorization or waiver.
	\glspar\hfill\textit{Source: Pilot/Controller Glossary, FAA, August 2025}}}

\newglossaryentry{classg}{
	name = {Class G airspace} ,
	description = {Airspace that is uncontrolled, except when associated with a temporary control tower, and has not been designated as Class A, Class B, Class C, Class D, or Class E airspace.
	\glspar\hfill\textit{Source: Pilot's Handbook of Aeronautical Knowledge, FAA, November 2023}}}

\newglossaryentry{controlled}{
	name = {controlled airspace} ,
	description = {An airspace of defined dimensions within which ATC service is provided to IFR and VFR flights in accordance with the airspace classification. It includes Class A, Class B, Class C, Class D, and Class E airspace.
	\glspar\hfill\textit{Source: Pilot's Handbook of Aeronautical Knowledge, FAA, November 2023}}}

\newglossaryentry{crewmember}{
	name = {crewmember (UAS)} ,
	description = {A person assigned to perform an operational duty. A UAS crewmember includes the remote pilot in command, the person manipulating the controls, and visual observers but may also include other persons as appropriate or required to ensure the safe operation of the UAS (e.g., sensor operator, ground control station operator).
	\glspar\hfill\textit{Source: Pilot/Controller Glossary, FAA, August 2025}}}

\newglossaryentry{mou}{
	name = {DoD / FAA Memorandum of Understanding (MoU)} ,
	description = {a /ac{mou} between the /ac{dod} and /ac{faa} that sets forth provisions that allow for increased /ac{dod} /ac{uas} use in the /ac{nas} outside of restricted, warning, or prohibited areas. This /ac{mou} defines many collaboration methods, and lists specific requirements that the /ac{dod} and /ac{faa} must meet.
	\glspar\hfill\textit{Source: /ac{mou} between the /ac{dod} and /ac{faa} for /{uas} in the /ac{nas}, Jointly published, May 2019}}}

\newglossaryentry{drone}{
	name = {drone} ,
	description = {See \term{uas}.
	\glspar\hfill\textit{Source: }}}

\newglossaryentry{flightcorridor}{
	name = {flight corridor} ,
	description = {A defined three-dimensional airspace, of defined width, which has been set aside for flights along a particular route.
	\glspar\hfill\textit{Source: ICAO Document 4444, International Civil Aviation Organization, 2016}}}

\newglossaryentry{flightpath}{
	name = {flight path} ,
	description = {The line, course, or track along which an aircraft is flying or is intended to be flown.
	\glspar\hfill\textit{Source: Pilot's Handbook of Aeronautical Knowledge, FAA, November 2023}}}

\newglossaryentry{geofence}{
	name = {geo-fence} ,
	description = {A virtual perimeter for a real-world geographic area that can be dynamically generated or correspond to a pre-set geographic boundary.
	\glspar\hfill\textit{Source: ASTM F3002-14a, ASTM International, 2014}}}

\newglossaryentry{groups}{
	name = {Group (\ac{dod} \ac{uas} Classification)} ,
	description = {The \ac{uas} group system establishes the foundation for joint \ac{uas} terminology. It provides a common reference to compare \ac{uas}. \ac{uas} are grouped based on the physical and performance characteristics of weight, operating altitude, and airspeed. \ac{uas} groups are determined without regard for payload, mission, command relationship, or Service. All \ac{uas} fall into one of five groups.
	\glspar\hfill\textit{Source: AR 95-1, DA, March 2018}}}

\newglossaryentry{group1}{
	name = {Group 1 \ac{uas}} ,
	description = {Typically hand-launched, self contained, portable systemsemployed for a small unit or base security. They are capable of providing “overthe hill” or “around the corner” reconnaissance and surveillance. They operatewithin visual range and are analogous to radio-controlled model airplanes ascovered in AC 91-57.30. Typically weigh less than 20lbs \ac{mgtow}.
	\glspar\hfill\textit{Source:  DoD UAS Airspace Integration Plan, 2004}}}

\newglossaryentry{group2}{
	name = {Group 2 \ac{uas}} ,
	description = {Small to medium in size and usually support brigade and intelligence, surveillance, reconnaissance, and target acquisition requirements. They usually operate from unimproved areas and launched via catapult. Payloads may include a sensor ball with electro-optic / infrared(EO/IR) and laser range finder/designator (LRF/D) capability. They typically perform special purpose operations or routine operations within a specific set of restrictions. Typically weigh between 21lbs and 55lbs \ac{mgtow}.
	\glspar\hfill\textit{Source:  DoD UAS Airspace Integration Plan, 2004}}}

\newglossaryentry{group3}{
	name = {Group 3 \ac{uas}} ,
	description = {Operate at medium altitudes with medium to long range and endurance. Their payloads may include a sensor ball with EO/IR, LRF/D,signal intelligence (SIGINT), communications relay, and chemical biological radiological nuclear explosive (CBRNE) detection. They usually operate from unimproved areas and may not require an improved runway. Typically weigh more than 55lbs, but less than 1320lbs \ac{mgtow}.
	\glspar\hfill\textit{Source:  DoD UAS Airspace Integration Plan, 2004}}}

\newglossaryentry{group4}{
	name = {Group 4 \ac{uas}} ,
	description = {Relatively large UAS that operate at medium to high altitudes and have extended range and endurance. They normally require improved areas for launch and recovery, beyond line-of-sight (BLOS) communications, andhave stringent airspace operations requirements. Payloads may include EO/IR sensors, radars, lasers, communications relay, SIGINT, Automatic Identification System (AIS), and weapons. Typically weigh more than 1320lbs \ac{mgtow}.
	\glspar\hfill\textit{Source:  DoD UAS Airspace Integration Plan, 2004}}}

\newglossaryentry{group5}{
	name = {Group 5 \ac{uas}} ,
	description = {Include the largest systems, operate at medium to high altitudes, and have the greatest range, endurance, and airspeed capabilities. They require improved areas for launch and recovery, BLOS communications, andthe most stringent airspace operations requirements. Group 5 UAS perform specialized missions such as broad area surveillance and penetrating attacks. Typically weigh more than 1320lbs \ac{mgtow}.
	\glspar\hfill\textit{Source:  DoD UAS Airspace Integration Plan, 2004}}}

\newglossaryentry{manipulator}{
	name = {manipulator of the controls} ,
	description = {A person who operates the controls of an aircraft during flight time.
	\glspar\hfill\textit{Source:  AC 61-65H, Federal Aviation Administration, September 2018}}}

\newglossaryentry{mgtow}{
	name = {Maximum Gross Takeoff Weight (MGTOW)} ,
	description = {The maximum allowable weight for takeoff, inclusive of airframe, fuel, and cargo.
	\glspar\hfill\textit{Source: Pilot's Handbook of Aeronautical Knowledge, FAA, November 2023}}}

\newglossaryentry{msl}{
	name = {mean sea level} ,
	description = {The average height of the surface of the sea at a particular location for all stages of the tide over a 19-year period.
	\glspar\hfill\textit{Source: Pilot's Handbook of Aeronautical Knowledge, FAA, November 2023}}}

\newglossaryentry{nas}{
	name = {National Airspace System (NAS)} ,
	description = {The common network of United States airspace—air navigation facilities, equipment and services, airports or landing areas; aeronautical charts, information and services; rules, regulations and procedures, technical information; and manpower and material.
	\glspar\hfill\textit{Source: Pilot's Handbook of Aeronautical Knowledge, FAA, November 2023}}}

\newglossaryentry{nsufr}{
	name = {National Security Unmanned Aircraft System Flight Restriction (NSUFR)} ,
	description = {Temporary flight restrictions that prohibit UAS operations within specified distances of certain security sensitive facilities to address national security concerns.
	\glspar\hfill\textit{Source: FAA UAS Data Exchange (LAANC), FAA, 2024}}}

\newglossaryentry{notam}{
	name = {Notice to Airmen (NOTAM)} ,
	description = {A notice filed with an aviation authority to alert aircraft pilots of any hazards en route or at a specific location. The authority in turn provides means of disseminating relevant NOTAMs to pilots.
	\glspar\hfill\textit{Source: Pilot's Handbook of Aeronautical Knowledge, FAA, November 2023}}}

\newglossaryentry{oop}{
	name = {Operations Over People (OOP)} ,
	description = {Operations of small unmanned aircraft over people.
	\glspar\hfill\textit{Source: Pilot/Controller Glossary, FAA, August 2025}}}

\newglossaryentry{operator}{
	name = {operator (UAS)} ,
	description = {The owner and/or remote pilot of a UAS.
	\glspar\hfill\textit{Source: Pilot/Controller Glossary, FAA, August 2025}}}

\newglossaryentry{part107}{
	name = {Part 107} ,
	description = {The Federal Aviation Regulation that establishes requirements for the operation of small unmanned aircraft systems (sUAS) in the National Airspace System.
	\glspar\hfill\textit{Source: 14 CFR Part 107, US Congress, August 2016}}}

\newglossaryentry{part91}{
	name = {Part 91} ,
	description = {General operating and flight rules that apply to aircraft operations within the United States and its territories.
	\glspar\hfill\textit{Source: 14 CFR Part 91, US Congress, April 2024}}}

\newglossaryentry{pilot}{
	name = {pilot} ,
	description = {The term “pilot” means an individual who has final authority and responsibility for the operation and safety of the flight or any other flight deck crew member.
	\glspar\hfill\textit{Source: 14 CFR Subsection 44921, US Congress, October 2018}}}

\newglossaryentry{pic}{
	name = {Pilot in Command (PIC)} ,
	description = {The pilot responsible for the operation and safety of an aircraft during flight time.
	\glspar\hfill\textit{Source: Pilot/Controller Glossary, FAA, August 2025}}}

\newglossaryentry{piloted}{
	name = {piloted} ,
	description = {Flight of an unmanned aircraft that allows pilot intervention in the management of the flight path.
	\glspar\hfill\textit{Source: ASTM F3269-17, ASTM International, 2017}}}

\newglossaryentry{rp}{
	name = {remote pilot} ,
	description = {Pilot of a UAS who is not operating as a recreational flyer under 49 USC §44809, the Exception for Limited Recreational Operations of Unmanned Aircraft.
	\glspar\hfill\textit{Source: Pilot/Controller Glossary, FAA, August 2025}}}

\newglossaryentry{rpic}{
	name = {Remote Pilot In Command (RPIC)} ,
	description = {The RPIC is directly responsible for and is the final authority as to the operation of the unmanned aircraft system.
	\glspar\hfill\textit{Source: Pilot/Controller Glossary, FAA, August 2025}}}

\newglossaryentry{tfr}{
	name = {Temporary Flight Restriction (TFR)} ,
	description = {Restriction to flight imposed in order to: 1. Protect persons and property in the air or on the surface from an existing or imminent flight associated hazard; 2. Provide a safe environment for the operation of disaster relief aircraft; 3. Prevent an unsafe congestion of sightseeing aircraft above an incident; 4. Protect the President, Vice President, or other public figures; and, 5. Provide a safe environment for space agency operations. Pilots are expected to check appropriate NOTAMs during flight planning when conducting flight in an area where a temporary flight restriction is in effect.
	\glspar\hfill\textit{Source: Pilot's Handbook of Aeronautical Knowledge, FAA, November 2023}}}

\newglossaryentry{uncontrolled}{
	name = {uncontrolled airspace} ,
	description = {Class G airspace that has not been designated as Class A, B, C, D, or E. It is airspace in which air traffic control has no authority or responsibility to control air traffic; however, pilots should remember there are VFR minimums which apply to this airspace.
	\glspar\hfill\textit{Source: Pilot's Handbook of Aeronautical Knowledge, FAA, November 2023}}}

\newglossaryentry{uav}{
	name = {Unmanned Aircraft (UA) / Unmanned Aerial Vehicle (UAV)} ,
	description = {A device used or intended to be used for flight that has no onboard pilot. This device can be any type of airplane, helicopter, airship, or powered-lift aircraft. Unmanned free balloons, moored balloons, tethered aircraft, gliders, and unmanned rockets are not considered to be a UA.
	\glspar\hfill\textit{Source: Pilot/Controller Glossary, FAA, August 2025}}}

\newglossaryentry{uas}{
	name = {Unmanned Aircraft System (UAS)} ,
	description = {An unmanned aircraft and its associated elements related to safe operations, which may include control stations (ground, ship, or air based), control links, support equipment, payloads, flight termination systems, and launch/recovery equipment. It consists of three elements: unmanned aircraft, control station, and data link
	\glspar\hfill\textit{Source: Pilot/Controller Glossary, FAA, August 2025}}}

\newglossaryentry{vlos}{
	name = {Visual Line of Sight (VLOS)} ,
	description = {Condition of operations wherein the operator maintains continuous, unaided visual contact with the unmanned aircraft
	\glspar\hfill\textit{Source: Pilot/Controller Glossary, FAA, August 2025}}}

\newglossaryentry{vo}{
	name = {visual observer} ,
	description = {A person who is designated by the remote pilot in command to assist the remote pilot in command and the person operating the flight controls of the small UAS (sUAS) to see and avoid other air traffic or objects aloft or on the ground.
	\glspar\hfill\textit{Source: Pilot/Controller Glossary, FAA, August 2025}}}

  

\AtBeginDocument{%
  % Show tooltip with the entry's description when hovering the term
  \NewDocumentCommand{\term}{m}{\pdftooltip{\gls{#1}}{\glsentrydesc{#1}}}
  \NewDocumentCommand{\Term}{m}{\pdftooltip{\Gls{#1}}{\glsentrydesc{#1}}}
  \NewDocumentCommand{\terms}{m}{\pdftooltip{\glspl{#1}}{\glsentrydesc{#1}}}
  \NewDocumentCommand{\Terms}{m}{\pdftooltip{\Glspl{#1}}{\glsentrydesc{#1}}}
}

\newglossarystyle{termabove}{%
  \setglossarystyle{list}% start from the basic list style
  \renewenvironment{theglossary}%
    {\begin{description}[leftmargin=0pt,labelsep=0pt,itemsep=.6\baselineskip]}%
    {\end{description}}%
  \renewcommand*{\glossentry}[2]{%
    \item[]%
      \glstarget{##1}{\textbf{\glsentryname{##1}}}\par
      \glossentrydesc{##1}\glspostdescription\space ##2%
  }%
  \renewcommand*{\glsgroupskip}{}% suppress A/B/C group headings
}

%%%%%%%%%%%%%%%%%%%%%%%%%%%%%%%%% % Reference  Formatting %%%%%%%%%%%%%%%%%%%%%%%%%%%%%%%%%

\crefformat{chapter}{Chapter~#2#1 (\descriptionlabel{#1})#3}
\Crefformat{chapter}{Chapter~#2#1 (\descriptionlabel{#1})#3}

%%%%%%%%%%%%%%%%%%%%%%%%%%%%%%%%% Paragraph Numbering %%%%%%%%%%%%%%%%%%%%%%%%%%%%%%%%%
% Your existing counter setup (keep as-is)
\setcounter{secnumdepth}{5}
\renewcommand{\thechapter}{\arabic{chapter}}
\renewcommand{\thesection}{\thechapter-\arabic{section}}

\makeatletter
\@addtoreset{paragraph}{section}
\@addtoreset{subparagraph}{paragraph}
\makeatother

\renewcommand{\theparagraph}{\alph{paragraph}}
\renewcommand{\thesubparagraph}{(\arabic{subparagraph})}

% Modified paragraph formatting with hanging indent
\titleformat{\paragraph}[runin]
  {\normalfont\normalsize}%
  {\textit{\theparagraph.}}{0.75em}{\bfseries}[\normalfont]

\titleformat{\subparagraph}[runin]
  {\normalfont\normalsize}%
  {\thesubparagraph}{0.75em}{\underline}[\normalfont]

% Modified spacing with hanging indent control
\titlespacing*{\paragraph}{1em}{0.5\baselineskip}{0.75em}
\titlespacing*{\subparagraph}{2em}{0pt}{0.75em} % Indent subparagraphs

%%%%%%%%%%%%%%%%%%%%%%%%%%%%%%%%% Document Information %%%%%%%%%%%%%%%%%%%%%%%%%%%%%%%%%

\title{Project MARS - Standard Operating Procedure}
\author{SFC Jamez White, Project NCOIC}
\date{01 Dec 2025}

%%%%%%%%%%%%%%%%%%%%%%%%%%%%%%%%% Document Content %%%%%%%%%%%%%%%%%%%%%%%%%%%%%%%%%
\begin{document}
\begin{titlepage}

\newcommand{\HRule}{\rule{\linewidth}{0.7mm}} % Defines a new command for the horizontal lines, change thickness here

\center % Center everything on the page
 
%----------------------------------------------------------------------------------------
%	HEADING SECTIONS
%----------------------------------------------------------------------------------------

\textsc{\LARGE Keller Army Community Hospital}\\[1cm] 

%----------------------------------------------------------------------------------------
%	LOGO SECTION
%----------------------------------------------------------------------------------------

\includegraphics[
    width=15cm,
    height=6cm,
    keepaspectratio,
    ]{KACH SOP/Project Mars Patch.png}\\[1cm] % Include a department/university logo - this will require the graphicx package

%----------------------------------------------------------------------------------------
%	TITLE SECTION
%----------------------------------------------------------------------------------------

\HRule \\[0.4cm]
{ \huge \bfseries Project MARS \\ Standard Operating Procedure}\\[0.4cm] % Title of your document
\HRule \\[1.5cm]
 
%----------------------------------------------------------------------------------------
%	AUTHOR SECTION
%----------------------------------------------------------------------------------------

\begin{minipage}{0.4\textwidth}
\begin{flushleft} \large
\textbf{Project OIC:}\\
LTC Samuel Teague % Your name
\end{flushleft}
\end{minipage}
~
\begin{minipage}{0.4\textwidth}
\begin{flushright} \large
\textbf{Project NCOIC:} \\
SFC Jamez White % Supervisor's Name
\end{flushright}
\end{minipage}\\[2cm]

% If you don't want a supervisor, uncomment the two lines below and remove the section above
%\Large \emph{Author:}\\
%John \textsc{Smith}\\[3cm] % Your name

%----------------------------------------------------------------------------------------
%	DATE SECTION
%----------------------------------------------------------------------------------------

{\vspace*{\fill}\large 01 December, 2025 \\\large\textbf{UNCLASSIFIED}}

%----------------------------------------------------------------------------------------

\vfill % Fill the rest of the page with whitespace

\end{titlepage}

\tableofcontents
\listoffigures
\listoftables

\chapter{General}
\label{ch:general}

\section{Purpose}
\paragraph{}This \ac{sop} establishes the operational procedures and operator selection/training requirements necessary to ensure safe flight operations for Project \ac{mars} by personnel within \ac{kach}. This \ac{sop} is designed to comply with all applicable \ac{faa}, \ac{dod}, and Army regulations, as well as \ac{kach} policies.

\section{Description of Operation}
\paragraph{}This \ac{sop} describes piloted and autonomous flight operations, maintenance and safety considerations, and personnel requirements related to the operation of multi-rotor \ac{suas} within the Project \ac{mars} inventory.
\\
\\
\noindent \textbf{Note:} This \ac{sop} must be present in the operational area of use. A copy will also be maintained in the \ac{kach} Project \ac{mars} office.

\section{Applicability}
This \ac{sop} applies to all government and contractor personnel who are actively involved in air or ground operations related to Project \ac{mars}.  Operational procedures described in this \ac{sop} will be used to maintain prudent, safe operating practices and to ensure that appropriate response actions are taken in the event of an emergency.

\section{Proponent and Authority}
\paragraph{Proponent}The proponent for this \ac{sop} is the \ac{kach} Project \ac{oic}.
\\
\paragraph{Exception/Waiver Authority}The proponent may approve written exceptions/waivers to this \ac{sop} that are consistent with controlling law, regulation, and safety policy. The proponent may delegate approval authority in writing to the Project \ac{ncoic}.
\\
\paragraph{Limits to Proponent Authority}Any change that alters \ac{opord}-controlled measures or authorities (for example, \ac{acm}s, altitude blocks, flight corridors/routes/boxes, \ac{tfb} approval lanes, or delegated command authority) requires action by the \ac{usma} G-3 via \ac{frago}.
\\
\paragraph{Recommended Changes}Users submit recommended changes via email to the Project \ac{ncoic} or Project \ac{oic}; the Project \ac{oic} adjudicates and publishes approved updates.
\\
\paragraph{Precendence}In the event of a conflict between this \ac{sop} and controlling law, regulation, or policy, the controlling document takes precedence. The proponent will correct the \ac{sop} at the next revision.

\chapter{Roles and Responsibilities}\label{ch:roles}

The Roles and Responsibilities of personnel involved in both the performance and oversight of Project \ac{mars} operations are defined in this chapter.

\section{Permanent \ Appointed Positions}

The positions listed below are permanent/appointed positions.  Appointment to positions specific to this Project will be accomplished by \ac{mfr} signed by the specified authority.

    \subsection{Hospital Commander}

        \paragraph{}
    
    \subsection{Project \ac{mars} \ac{oic}}

            \paragraph{}
            \paragraph{Appointment} The Project \ac{mars} \ac{oic} will be appointed by the Hospital Commander via MFR.

    \subsection{Project \ac{mars} \ac{ncoic}}

            \paragraph{}
            \paragraph{Appointment} The Project \ac{mars} \ac{ncoic} will be appointed by the Hospital Commander via MFR.
    
    \subsection{Lead Operator}

            \paragraph{}
            \paragraph{Appointment} The Project \ac{mars} Lead Operator will be appointed by the Project \ac{mars} \ac{oic} via MFR.
    
    \subsection{Project \ac{mars} Technical Lead}
    
        \paragraph{} A government employee within \ac{kach} will be designated the Project \ac{mars} Technical Lead.  This individual is responsible for the general configuration management and maintenance of the Project \ac{mars} fleet. The Primary Operator is responsible for managing all hardware and software configurations of the Project  \ac{mars} fleet in accordance with the \ac{awr} issued by \ac{srd}.  
        
        \paragraph{}The Project \ac{mars} is also responsible for training and approving other technicians.
        \paragraph{Appointment} The Project \ac{mars} Technical Lead will be appointed by the Project \ac{mars} \ac{oic} via MFR.
    
    \subsection{Maintenance Technician}
    
        \paragraph{} Maintenance will typically be performed by the Project \ac{mars} Technical Lead.  Maintenance technicians responsible for conducting all maintenance and upkeep required to ensure the safe operation of Project \ac{mars} vehicles. Personnel will authorize the use of each vehicle in the Project \ac{mars} based on completion of appropriate inspection(s).  
        
        \paragraph{}Maintenance Technicians will be available on request to assist \ac{rpic}s in setting up the Project \ac{mars} for flight and conducting the preflight inspection as necessary. 
        
        \paragraph{}Communications related to safety will be conducted with the Project \ac{mars} NCOIC and with the Project \ac{mars} Technical Lead.  In the event of an emergency, maintenance technicians will stand by for instructions from the Project \ac{mars} \ac{ncoic} in recovery of the air vehicle.
        
        \paragraph{Appointment} Maintenance Technicians will be appointed by the Project \ac{mars} Technical Lead via MFR.

    \subsection{Chief of Pharmacy}

            \paragraph{}


\section{Situational Positions}

    \subsection{\acf{rpic}}

            \paragraph{}
            \paragraph{Qualification}

    
    \subsection{Crewmember}

            \paragraph{}
            \paragraph{Qualification} At a minimum, all assigned crewmembers will have an \ac{ao} qualification. Requirements for this qualification are laid out in \Cref{ch:certification}.

    \subsection{\acf{vo}}

            \paragraph{}
            \paragraph{Qualification}

    
    

\chapter{Airframe \& Supporting Systems}
\section{Approved Airframes}
\section{Configuration Management and Documentation}
\section{System Specifications}
\subsection{Airframe Performance}
\subsection{Command and Control (C2) Systems}
\subsection{Beyond Visual Line of Sight (BVLOS) Systems}
\section{Cargo Approved for Flight}
\section{Ground Control Station (GCS) and Software}
\chapter{Pilot Certification, Training, \& Currency}
\label{ch:certification}

        \subsubsection{}This chapter establishes the \acf{atp} for Project \ac{mars}.  Project \ac{mars}' primarily operates nonstandard, \ac{uas}s with a non-tactical mission, and its qualification, evaluation, and currency requirements are defined in AR 95-1, Appendix D, Paragraph 10.  The Project \ac{mars} \ac{atp} provides specific guidelines for executing \ac{suas} aircrew training, and establishes crewmember qualification, refresher, mission, and continuation training and evaluation requirements.

        \subsubsection{} The operator’s manual is the governing authority for operation of the aircraft. If differences exist between the maneuver descriptions in the operators’ manual and this chapter, then this chapter the governing authority for training and flight evaluation purposes only. 

    \section{Operator Qualification Framework}

            \subsubsection{} The Project \ac{mars} operator qualification framework is a tiered system designed to ensure personnel possess the appropriate skills and knowledge for the missions they are assigned. Advancement through these tiers is based on a structured progression of training, demonstrated proficiency, and a comprehensive understanding of program procedures. This framework ensures that while all operators meet a baseline safety standard, those tasked with more complex or higher-risk missions have achieved a corresponding level of expertise.

            \subsubsection{} The three levels of qualification are \acf{ao} , \acf{qp} , and \acf{eo} . 

                \paragraph{\acl{ao}} An \acl{ao} represents the entry-level qualification, trained specifically to oversee and manage pre-planned, routine autonomous missions under normal conditions. 
\\
               \paragraph{\acl{qp}}The next level, \acl{qp} , builds upon that foundation by adding comprehensive manual flight proficiency. A \ac{qp} is skilled in taking direct manual control of the aircraft, especially during off-nominal situations, confined area landings, or when deviating from an autonomous plan. 
\\
                \paragraph{\acl{eo}}The highest level of qualification is the \acl{eo} , an individual with extensive experience who serves as an instructor, evaluator, and subject matter expert. \ac{eo}s are responsible for conducting training, certifying other operators, and leading the development and validation flights for new routes and emergency procedures.

    \section{Requirements for All Operators}
    
            \subsubsection{} The following personnel may fly and/or operate Project \ac{mars} airframes.  Operators who --
\\
                    \paragraph{} Are members of the Regular Army, \ac{usar}, \ac{arng}, or Civilian employees of the U.S. Army or \ac{dha}.
\\
                    \paragraph{} Have complied with the qualification, training, evaluation, and currency requirements of this SOP for the airframe to be flown and/or operated.
\\
                    \paragraph{} Meet the medical standard as outlined in AR 40-501 (but are not required to maintain a class IV physical).
\\
    \section{\acf{ao} Qualification}
    
            \subsubsection{} \acf{ao} level is the most basic operator certification level, and consists of \ac{buq} 1 and 2 certification completed via the SUASMAN website (or equivalent), plus 2 hours local training on the Project \ac{mars} airframe.  This level certifies the pilot to operate the \ac{kach} airframe in autonomous mode  only, under supervision of a higher-level Pilot Evaluator.
        
            \subsubsection{}Personnel trained to \ac{ao} Level are required to recertify with a currently qualified Pilot Evaluator annually. 

    \section{\acf{qp} Qualification}
   
    \section{\acf{eo} Qualification}
  
    \section{Instructor Pilot (IP) Training / Pilot Evaluator Training}
   
    \section{Currency and Recurrency Requirements}
\chapter{Flight Operations}
\label{ch:flightops}

\section{Types of Operations}
    \subsection{Routine Scheduled Delivery}
    \subsection{On-Demand / Emergency Response Delivery}
    \subsection{\ac{dte} Flights}
    \subsection{Training Flights}
\section{Air Corridor and Route Management}
\section{Mission Planning and Approval Cycle}
\section{Flight Procedures}
\section{Data Management}
\chapter{Maintenance and Logistics}
\label{ch:maintenance}

\section{Maintenance Overview}

\section{Maintenance Schedule}

\section{Battery Management Protocol}

\section{Software and Firmware Update Procedures}

\section{Supply and Spare Parts Management}

\section{Maintenance Documentation}
\chapter{Safety and Emergency Procedures}
\label{ch:safety}

    \section{Risk Management}

        \subsection{Controlling Doctrine}Risk management for \projectmars{} is done in accordance with DA Pamphlet 385-30. Additionally, all \projectmars{} \ac{suas} will have current \ac{awr}s in accordance with AR 70-62 prior to conducting flight operations. 

        \subsection{Risk Acceptance Authority} The overall \ac{raa} for \projectmars{} is the \nameref{role:hospitalcommander}. The \nameref{role:hospitalcommander} has delegated \ac{raa} to the \nameref{role:projectoic} for nonrecurring missions conducted under \ac{vlos} conditions. Because of the potential risk to non-participating personnel during \ac{bvlos} operations, the \ac{raa} for \ac{bvlos} operations is the \ac{smc}, or their delegate. The \ac{usma} Superintendent, as \ac{smc} has delegated this approval authority to the \ac{wp} Garrison Commander.

        \subsection{Training Flight Risk}All \projectmars{} training flights are considered low risk since the aircraft must be operated within line of sight of the \nameref{role:rpic} or \nameref{role:visualobserver}.  Training flights are restricted to unpopulated areas where risk to people and property is minimal. During these flights, line of sight will be maintained between the aircraft and ground station at all times.

        \subsection{}All \projectmars{} flights conducted for other than training must have a hazard analysis and a risk determination. \projectmars{} Personnel will obtain briefings from the \nameref{role:projectncoic} and implement the mitigations in accordance with the risk determination prior to any flight operations. Briefings will be conducted by the /ac{rpic}.

        \subsection{}
        
    \section{General Emergency Procedures}\label{sec:ep-general}
    
            \subsubsection{Priorities} This \ac{sop} cannot anticipate every type of emergency that an operation may encounter. Any \acf{ep} laid out in this chapter for specific emergencies can be equated to 'immediate action' techniques for a small arms weapons system. They do not replace critical thinking. \projectmars{} personnel will, at all times, prioritize the following criteria in order:
            
                \paragraph{Preservation of Life} No stateside operation is so important that it is worth the loss of a single human life.  At all times, \projectmars{} personnel will act to minimize the risk of injury/fatality to anyone, whether or not they are directly involved with the Operation.
                
                \paragraph{Preservation of Materiel} While \ac{suas} are considered consumable commodities, rather than durable property, they are non-trivial to replace and their safety should be prioritized when possible. In an emergency, \projectmars{} personnel will prioritize safeguarding non-\projectmars{} materiel ahead of \ac{suas} equipment.
                
                \paragraph{Preservation of Mission} The mission of \projectmars{} is to develop a safe, reliable, autonomous \ac{suas} based resupply capability. However, no individual operation is so critical that its accomplishment overrides general safety procedures. In an emergency, \projectmars{} personnel will prioritize mission accomplishment only after ensuring that there is no further risk to personnel or materiel; if any doubt exist, the mission will be scrubbed.
    
            \subsubsection{Pilot Priorities} Loss of control of the aircraft remains the highest risk related to sUAS operations. The following priorities should be followed by operators during an in-flight emergency:
    
                    \paragraph{Aviate} When an aircraft is in the air, the top priority of the \ac{rpic} must always be to aviate. That means to maintain (or regain) manual control of the aircraft.
    
                    \paragraph{Navigate} After ensuring they have control of the aircraft, the \ac{rpic} must navigate the aircraft to the safest location possible in accordance with the mission parameters. \ac{rpic}'s must keep in mind that while this may be either \ac{lz}, or \ac{saa}, if the aircraft is unable to reach any of those locations safely, the \ac{rpic} must use their best judgement to land the aircraft in the safest manner possible considering the above priorities.
    
                    \paragraph{Communicate}  Once the \ac{rpic} has positive control of the aircraft they should communicate the issue appropriately as laid out in this \ac{sop}. This may be done simultaneous to the navigate step above, but should not take priority over safe navigation or operation of the aircraft.
            
    \section{Specific Emergency Procedures}

            \subsubsection{} The following specific \ac{ep}s reflect the most common emergencies encountered by \ac{suas} operators. These \ac{ep}s should be followed to the extent that they align with the priorities listed in \cref{sec:ep-general}.

        \subsection{Lost Communication (Lost Link)}\label{ssec:ep-losslink}
           
            \subsubsection{Situation}Lost Link refers to the loss of \ac{c2} between the \ac{gcs} and the \ac{suas} during flight. This includes primary and backup telemetry, command uplink, or video downlink loss beyond a pre-established threshold. Lost link may be temporary or permanent and can occur in both \ac{vlos} and \ac{bvlos} operations. All aircraft participating in \projectmars{} flights must be pre-programmed with an appropriate Lost Link procedure and contingency route.
        
            \subsubsection{Detection and Initial Assessment} Lost Link is detected when:
                
                \paragraph{} The \ac{gcs} indicates telemetry or control signal interruption (e.g., "No RC Signal," "Lost Telemetry").
                    
                \paragraph{} The aircraft ceases to respond to manual inputs.

                \paragraph{} A pre-set timeout value (e.g., 3 seconds for control link, 10 seconds for telemetry) is exceeded without recovery.
    
            \subsubsection*{}The Operator shall immediately verify the condition by confirming:\setcounter{paragraph}{0}
            
                \paragraph{}Power and antenna integrity at the \ac{gcs}.
                    
                \paragraph{}System status on secondary displays or via secondary control links (if available).
                    
                \paragraph{}Possible sources of local interference or terrain obstruction.
                    
            \subsubsection{Priorities}
        
                \paragraph{}Ensure safety of personnel, airspace users, and  infrastructure.

                \paragraph{}Prevent airspace incursion outside the authorized corridor.

                \paragraph{}Recover the aircraft safely to a pre-designated Rally Point or \ac{rth} location.

                \paragraph{}Maintain accountability of payloads.

                \paragraph{}Comply with \ac{faa}, \ac{dod}, and \ac{kach} safety policies.
                
            \subsubsection{Immediate Actions (All Flights)}

               \paragraph{}Attempt manual link recovery within 30 seconds.

                \paragraph{}If unsuccessful, monitor aircraft telemetry (if still available) and observe for programmed Lost Link behavior:
                    
                    \subparagraph{}Autonomous \ac{rth}
                
                    \subparagraph{}Loiter-and-Land
                    
                    \subparagraph{} Pre-programmed contingency route
                    
                \paragraph{} Notify XXX

                \paragraph{} If aircraft is non-responsive and on unsafe trajectory, initiate termination protocol.

                \paragraph{} Log time, location, telemetry status, and any deviation from planned route.
                                
            \subsubsection{\ac{bvlos} Specific Considerations}
      
            \subsubsection{\ac{vlos} Specific Considerations}
       
            \subsubsection{Reporting and Documentation}
       
            \subsubsection{Maintenance and Return to Service}
            
        \subsection{GPS/Navigation Failure}\label{ssec:ep-gpsfail}
        
            \subsubsection{Situation} \ac{gps} / Navigation Failure describes loss or severe degradation of onboard positioning data in primarily autonomous flight. When satellite signals, inertial sensors, or magnetometers become unreliable, the aircraft may revert to inertial dead reckoning, drift off the programmed track, or enter conservative failsafe behavior without human intervention. Operators must be prepared to monitor telemetry, switch to alternate navigation modes, and (if available) prompt the aircraft toward a safe loiter or landing using minimal manual inputs.

            \subsubsection{Detection and Initial Assessment}
            
            \subsubsection{Priorities}
            
            \subsubsection{Immediate Actions (All Flights)}
            
            \subsubsection{\ac{bvlos} Specific Considerations}
            
            \subsubsection{\ac{vlos} Specific Considerations}
            
            \subsubsection{Reporting and Documentation}
            
            \subsubsection{Maintenance and Return to Service}
        
        \subsection{Uncommanded Flight (Flyaway)}\label{ssec:ep-flyaway}
        
            \subsubsection{Situation}Uncommanded Flight (Flyaway) refers to autonomous aircraft motion that diverges from the programmed mission without human input. Control system anomalies, corrupt waypoints, sensor faults, or malicious interference can cause unexpected climbs, turns, or speed changes while automation remains engaged. Crew members must quickly recognize abnormal behavior, assess whether the autopilot remains responsive to high-level overrides, and be prepared to transition the platform to stabilized manual control or initiate termination protocols if containment fails.

            \subsubsection{Detection and Initial Assessment}
            
            \subsubsection{Priorities}
            
            \subsubsection{Immediate Actions (All Flights)}
            
            \subsubsection{\ac{bvlos} Specific Considerations}
            
            \subsubsection{\ac{vlos} Specific Considerations}
            
            \subsubsection{Reporting and Documentation}
            
            \subsubsection{Maintenance and Return to Service}
            
        \subsection{Intrusion of Local Airspace}\label{ssec:ep-intrusion}
        
            \subsubsection{Situation}Intrusion of Local Airspace occurs when an uncooperative aircraft enters the \ac{rwc} volume around an autonomously flown \ac{suas}. Unexpected manned or unmanned traffic may appear without prior coordination, penetrate the \ac{rwc} volume from any direction, and present an immediate midair collision risk while automation remains engaged. Operators must watch for \ac{daa} or geofence alerts, issue prompt reroute or hold commands to increase separation, and be prepared to assume manual control or execute termination protocols if collision risk cannot be mitigated.

            \subsubsection{Detection and Initial Assessment}
            
            \subsubsection{Priorities}
            
            \subsubsection{Immediate Actions (All Flights)}
            
            \subsubsection{\ac{bvlos} Specific Considerations}
            
            \subsubsection{\ac{vlos} Specific Considerations}
            
            \subsubsection{Reporting and Documentation}
            
            \subsubsection{Maintenance and Return to Service}
        
        \subsection{Low Battery Emergency Landing}\label{ssec:ep-lowbattery}
        
            \subsubsection{Situation}Low Battery Emergency Landing is triggered when an autonomously flown aircraft forecasts insufficient power to complete the programmed mission safely. Accelerated discharge, cell imbalance, or temperature-induced capacity loss may reduce available endurance without human inputs to conserve energy. The automation may command aggressive power-saving modes, altitude reductions, or immediate descent; operators should supervise these transitions, select the nearest safe landing area, and provide minimal manual guidance only as needed to ensure a controlled touchdown before critical levels are reached.

            \subsubsection{Detection and Initial Assessment}
            
            \subsubsection{Priorities}
            
            \subsubsection{Immediate Actions (All Flights)}
            
            \subsubsection{\ac{bvlos} Specific Considerations}
            
            \subsubsection{\ac{vlos} Specific Considerations}
            
            \subsubsection{Reporting and Documentation}
            
            \subsubsection{Maintenance and Return to Service}
            
    \section{Accident/Incident Reporting}\label{sec:accidentreporting}

        \subsubsection{} Standard operations of the \ac{uas} may involved flight termination within the operating area. Nominal operations may result in mid-air collisions between \projectmars{} \ac{uas} within the operating area, which may cause components to break. These events do not constitute mishaps that are reportable outside the \projectmars{} chain of command. Reportable mishaps include:

            \paragraph{}Flight outside approved airspace

            \paragraph{}\ac{suas} operations leading to injury

            \paragraph{}\ac{suas} operations leading to damge to property not invovled with \projectmars{}.
        
        \subsubsection{} In the event of a reportable mishap, crash, vehicle accident, or injury all \projectmars{} \ac{suas} are immediately grounded until cleared by the \namecref{role:projectoic} via \ac{mfr}.
        
        \subsubsection{\acf{rpic} Responsibilities}In the event of a \ac{uas} mishap, vehicle accident, or injury, the following steps will be taken by the \ac{rpic}:

            \paragraph{}Stop the mission.

            \paragraph{}In the event of injury to personnel, or ongoing threat to life or property (e.g. fire), dial 911.
            
            \paragraph{}Contact \namecref{role:projectncoic} about incident; if unable to reach \namecref{role:projectncoic}, escalate to \namecref{role:projectoic}.  If unable to reach either the \namecref{role:projectncoic} or \namecref{role:projectoic} and there are injuries, make contact with the \namecref{role:hospitalcommander}

            \paragraph{}Collect the following information:

                \subparagraph{}Type of \ac{uas}

                \subparagraph{}Model of \ac{uas}

                \subparagraph{}Type of Assistance Needed

                \subparagraph{}Location of incident

                \subparagraph{}Type and severity of injuries

                \subparagraph{}Names of injured

                \subparagraph{}Date and time

                \subparagraph{}Personnel and property involved in accident

                \subparagraph{}any additional hazards remaining (eg, fire/hazmat/etc)

            \paragraph{}Apply the following precautions:

                \subparagraph{}Keep others away for their own safety.

                \subparagraph{}Render first aid.

                \subparagraph{}Secure and control the accident site.

                \subparagraph{}Advise personnel help is on the way. 

                \subparagraph{}Remain at accident site until relieved either by \namecref{role:projectncoic} or \namecref{role:projectoic}.

        \subsubsection{\namecref{role:projectncoic} Responsibilities} Upon notification of a mishap, the \namecref{role:projectncoic} will:

            \paragraph{}Immediately move to the incident site to provide supervision.

            \paragraph{}Notify \namecref{role:projectoic} of incident. 

            \paragraph{}Ensure all aircraft and crewmember flight records are secured. 

            \paragraph{}Recover the aircraft when safe.

            \paragraph{}Be prepared to brief the \namecref{role:hospitalcommander} as requested.

            \paragraph{}Provide resources and assistance to accident investigators as necessary.

        \subsubsection{\namecref{role:projectoic} Responsibilities} Upon notification of a mishap, the \namecref{role:projectoic} will:

            \paragraph{}Notify the \namecref{role:hospitalcommander} as appropriate.

            \paragraph{}Identify the nature of the mishap and appoint appropriate personnel to investigate. In the event of injury, damage to non-Project material, and in other circumstances the \ac{oic} deems appropriate, the \namecref{role:projectoic} will recommend appointment of \ac{io} to the chain of command.

            \paragraph{}Review findings and recommendations of \ac{io}, if appointed, or \namecref{role:projectncoic} and record course of action and approval for resumption of flight operations via \ac{mfr}.

\begin{appendices}
\chapter{Regulatory Framework and References}
\chapter{Checklists}
\section{Pre-Flight Checklist}
\section{Post-Flight Checklist}
\section{Emergency Procedures Checklist}
\chapter{Required Forms}
\section{Pre-Flight Inspection Form}
\section{Flight Log Form}
\section{Maintenance Log Form}
\section{Incident Report Form}

\chapter{Approved \ac{uas} Characteristics}\label{ch:approved_uas}

% Requires: \usepackage{booktabs}
The following \ac{uas} have been approved for operations as part of \projectmars, as laid out in \claudereplace{ref{ch:airspace}}{Cref{ch:airspace}}. Configurations are subject to change based on mission requirements and technological advancements. Operators must ensure compliance with all relevant regulations and guidelines when utilizing these \acp{uas}.

    \section{Airframes approved for \ac{vlos} operations}

    \section{Airframes approved for \ac{bvlos} operations}

\end{appendices}
\include{glossary.tex}
\end{document}
