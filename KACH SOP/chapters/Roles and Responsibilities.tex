\chapter{Roles and Responsibilities}
\label{ch:roles}

            \subsubsection{} The Roles and Responsibilities of personnel involved in both the performance and oversight of \projectmars{} operations are defined in this chapter.

    \section{Permanent \ Appointed Positions} The positions listed below are permanent/appointed positions. Appointment to positions specific to this Project will be accomplished by \ac{mfr} signed by the specified authority.

        \subsection{Hospital Commander}\label{role:hospitalcommander}

            \subsubsection{}
    
        \subsection{Project MARS OIC}\label{role:projectoic}

            \subsubsection{Primary Role}A US Army Officer will be designated the \projectmars{} \ac{oic}. The \ac{oic} is responsible for oversight of the Project, and is empowered to make decisions regarding the planning, briefing, and execution of all flight operations. The \ac{oic} will be the principal advisor to the \nameref{role:hospitalcommander} regarding the Project. 
            
            \subsubsection{Responsibilities} The \ac{oic}'s duties include, but are not limited to the following:

            \begin{itemize}
                
                \item Assure that an assessment of the operational area meets any requirement prescribed for that location in accordance with FAA and/or Army guidance and policy.
                
                \item All material hazards have been identified and eliminated or adequate mitigations have been implemented prior to flight operations.
                
                \item A risk determination for the material hazards has been made and all risks assessed and accepted at the appropriate level in a signed risk acceptance memorandum to be provided to SRD prior to flight operations.
                
                \item All operational hazards have been identified and eliminated or adequate mitigations have been implemented prior to flight operations.
                
                \item A risk determination for the operational area hazards has been made and all risks assessed and accepted at the appropriate level in a signed risk acceptance memorandum to be provided to SRD prior to flight operations.

                \item Supervision for all personnel involved in the mission.

                \item Ensuring all personnel supporting the operation are qualified for the role they are assigned, have been properly briefed and thoroughly understand their role in the mission.

                \item Logbooks are present and accurate prior to all flight operations.

                \item All system equipment is verified in proper working order during the preflight inspection and properly set-up.

                \item The mission data is recorded in the logbook.

                \item Ensures that all required communications with the other controlling agencies are established and maintained throughout the entire operation.

                \item A Mishap Action plan is in place prior to operations for each operational area with emergency notification procedures completed prior to flight.

                \item Conducts pre/post mission briefs with all personnel involved in the mission as prescribed within this SOP.
            \end{itemize}
            
            \subsubsection{Appointment} The \projectmars{} \ac{oic} will be appointed by the \nameref{role:hospitalcommander} via MFR.

        \subsection{Project MARS NCOIC}\label{role:projectncoic}

            \subsubsection{Primary Role}

            \subsection{Additional Responsibilities}
            
            \subsubsection{Appointment} The \projectmars{} \ac{ncoic} will be appointed by the \nameref{role:hospitalcommander} via MFR.
    
        \subsection{Lead Operator}\label{role:leadoperator}

            \subsubsection{Primary Role} The Lead Operator is the \projectmars{} \ac{sme} when it comes to flight operations., and will be \ac{eo} qualified. The Lead operator helps the \nameref{role:projectoic} develop,implement, and manage the \ac{atp}. The Lead Operator will conduct \ac{iqt} and \nameref{sec:qual-ao}.  If \ac{suas} Master Trainer Qualified, the Lead Operator will conduct \nameref{sec:qual-qp}; if not master qualified, the Lead Operator will assist with training of \projectmars{} personnel selected for \ac{qp} training.

            \subsubsection{Additional Responsibilties} The Lead Operator will be responsible for maintaining the flight qualification training records for \projectmars{} personnel.
            
            \subsubsection{Appointment} The \projectmars{} Lead Operator will be appointed by the \projectmars{} \ac{oic} via MFR.
    
        \subsection{Project MARS Technical Lead}\label{role:technicallead}
    
            \subsubsection{Primary Role} A government employee within \ac{kach} will be designated the \projectmars{} Technical Lead. This individual is responsible for the general configuration management and maintenance of the \projectmars{} fleet. The Technical Lead is responsible for managing all hardware and software configurations of the \projectmars{} fleet and certifying individual airframes for flight in accordance with the \ac{awr} issued by \ac{srd}. Communications related to safety of the airframe will be directed toward the Technical Lead.
        
            \subsubsection{Additional Responsibilities}The \projectmars{} Technical Lead is also responsible for training and approving maintenance technicians.
               
            \subsubsection{Appointment} The \projectmars{} Technical Lead will be appointed by the \projectmars{} \ac{oic} via MFR.
    
        \subsection{Maintenance Technician}\label{role:maintenancetech}
    
            \subsubsection{Primary Role} Maintenance technicians are responsible for conducting all maintenance and delegated to them by the \nameref{role:technicallead}, as well as any upkeep required to ensure the safe operation of \projectmars{} vehicles. Personnel will certify the use of each vehicle in the \projectmars{} based on completion of appropriate inspection(s).
        
            \subsubsection{Additional Responsibilities}Maintenance Technicians will be available on request to assist \ac{rpic}s in setting up the \projectmars{} for flight and conducting the preflight inspection as necessary. In the event of an emergency, maintenance technicians will stand by for instructions from the \projectmars\nameref{role:projectncoic} for directions related to recovery operations.
        
            \subsubsection{Appointment} Maintenance Technicians will be appointed by the \nameref{role:technicallead} via MFR.

    \section{Situational Positions}  The following positions are not permanent positions and are assigned on a per-flight basis. 

        \subsection{\acf{rpic}}\label{role:rpic}

            \subsubsection{}A designated government employee is assigned as the \ac{rpic} for all flights. The \ac{rpic} is responsible for evaluating the situation and making timely decisions for the safe operation of the \projectmars{} aircraft at all times, even when the aircraft is following pre-programmed commands. The \ac{rpic} is the final authority as to the operation of the aircraft. The \ac{rpic}'s duties include, but are not limited to:

                \paragraph{Pre-flight} Prior to flight, the \ac{rpic} is responsible for: 
                
                \begin{itemize}
                
                    \item Confirming Flight authorization with \nameref{role:projectncoic}. 

                    \item Confirming airworthiness certification of the aircraft to be flown, and conducting pre-flight inspection IAW \Cref{ssec:preflight}.

                    \item Receiving cargo and ensuring it is stowed in the payload securely and that it will not adversely affect the flight characteristics or controllability of the aircraft. (See \Cref{sec:approvedcargo}.)

                    \item Confirming that weather conditions are safe for flight.

                    \item Confirming that a \ac{notam} is published, and there are no flight restrictions in place that would ground the flight.

                    \item Ensuring all \nameref{role:crewmember}s are properly briefed on operating conditions, contingency procedures, and emergency procedures.

                    \item Making verbal connection with the \nameref{role:crewmember}(s) on the arriving \ac{lz} and ensure they are staged and prepared to receive the aircraft BEFORE liftoff.
                    
                \end{itemize}

                \paragraph{During Flight} During flight, the \ac{rpic} is responsible for: 

                    \subparagraph{All Flights} The \ac{rpic}  is responsible for complying with all procedures and restrictions outlined in this \ac{sop}, and any applicable \ac{coa}. In the event of an emergency, the \ac{rpic} will maintain/regain control of the aircraft to the best of their ability and bring the aircraft to a safe \ac{lz} or the nearest \ac{saa} designated for that flight. The \ac{rpic} has final decision on when to land or ditch the aircraft, but may not violate /ac{faa} or  Army regulations in an effort to "save" the aircraft. 
                    
                    \subparagraph{Manual Flight} During manual flight, the \ac{rpic}'s primary task is to maintain control of the aircraft. Manual flights will usually be conducted in \ac{vlos} conditions, and the \ac{rpic} will ensure that they or the \ac{vo} has unaided sight of the aircraft at all times. 

                    \subparagraph{Autonomous Flight} During autonomous flight, the \ac{rpic} is responsible for monitoring the aircraft's telemetry at all times and must remain ready to directly intervene if necessary. The \ac{rpic} will maintain communication throughout the flight with the receiving \nameref{role:crewmember}(s) to ensure everyone knows what the other \nameref{role:crewmember}s are doing, and who is controlling the aircraft.
                    
                \paragraph{Post-Flight} After a flight, the \ac{rpic} is responsible for: 
                
                    \begin{itemize}
                    
                        \item Confirming that the aircraft has arrived safely to the receiving \ac{lz} and has been secured by receiving \nameref{role:crewmember}(s).

                        \item Reporting successful flight to \nameref{role:projectncoic}, and filing any requisite flight logs or other paperwork associated with the flight.

                        \item Reporting any issues, irregularities, malfunctions, or mishaps that occurred to the \nameref{role:leadoperator}, \nameref{role:technicallead}, and/or \nameref{role:projectncoic} as appropriate. 
                        
                    \end{itemize}
                    
            \subsubsection{Qualification} Qualification requirements to serve as \ac{rpic} are laid out in \Cref{ch:certification}. At a minimum, personnel service as an \ac{rpic} for autonomous flights will be certified to \nameref{sec:qual-ao} standards. To serve as \ac{rpic} during manual flight modes, personnel will be certified to \nameref{sec:qual-qp} standards.
    
        \subsection{Crewmember}\label{role:crewmember}

            \subsubsection{}
            
            \subsubsection{Qualification} At a minimum, all assigned crewmembers will have an \ac{ao} qualification. Requirements for this qualification are laid out in \cref{ch:certification}.

        \subsection{\acf{vo}}\label{role:visualobserver}

            \subsubsection{}
            
            \subsubsection{Qualification}

    
    
