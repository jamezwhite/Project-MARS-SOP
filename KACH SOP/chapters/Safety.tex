\chapter{Safety and Emergency Procedures}
\label{ch:safety}

    \section{Risk Management}

        \subsection{Controlling Doctrine}Risk management for \projectmars{} is done in accordance with DA Pamphlet 385-30. Additionally, all \projectmars{} \ac{suas} will have current \ac{awr}s in accordance with AR 70-62 prior to conducting flight operations. 

        \subsection{Risk Acceptance Authority} The overall \ac{raa} for \projectmars{} is the \nameref{role:hospitalcommander}. The \nameref{role:hospitalcommander} has delegated \ac{raa} to the \nameref{role:projectoic} for nonrecurring missions conducted under \ac{vlos} conditions. Because of the potential risk to non-participating personnel during \ac{bvlos} operations, the \ac{raa} for \ac{bvlos} operations is the \ac{smc}, or their delegate. The \ac{usma} Superintendent, as \ac{smc} has delegated this approval authority to the \ac{wp} Garrison Commander.

        \subsection{Training Flight Risk}All \projectmars{} training flights are considered low risk since the aircraft must be operated within line of sight of the \nameref{role:rpic} or \nameref{role:visualobserver}.  Training flights are restricted to unpopulated areas where risk to people and property is minimal. During these flights, line of sight will be maintained between the aircraft and ground station at all times.

        \subsection{}All \projectmars{} flights conducted for other than training must have a hazard analysis and a risk determination. \projectmars{} Personnel will obtain briefings from the \nameref{role:projectncoic} and implement the mitigations in accordance with the risk determination prior to any flight operations. Briefings will be conducted by the /ac{rpic}.

        \subsection{}
        
    \section{General Emergency Procedures}\label{sec:ep-general}
    
            \subsubsection{Priorities} This \ac{sop} cannot anticipate every type of emergency that an operation may encounter. Any \acf{ep} laid out in this chapter for specific emergencies can be equated to 'immediate action' techniques for a small arms weapons system. They do not replace critical thinking. \projectmars{} personnel will, at all times, prioritize the following criteria in order:
            
                \paragraph{Preservation of Life} No stateside operation is so important that it is worth the loss of a single human life.  At all times, \projectmars{} personnel will act to minimize the risk of injury/fatality to anyone, whether or not they are directly involved with the Operation.
                
                \paragraph{Preservation of Materiel} While \ac{suas} are considered consumable commodities, rather than durable property, they are non-trivial to replace and their safety should be prioritized when possible. In an emergency, \projectmars{} personnel will prioritize safeguarding non-\projectmars{} materiel ahead of \ac{suas} equipment.
                
                \paragraph{Preservation of Mission} The mission of \projectmars{} is to develop a safe, reliable, autonomous \ac{suas} based resupply capability. However, no individual operation is so critical that its accomplishment overrides general safety procedures. In an emergency, \projectmars{} personnel will prioritize mission accomplishment only after ensuring that there is no further risk to personnel or materiel; if any doubt exist, the mission will be scrubbed.
    
            \subsubsection{Pilot Priorities} Loss of control of the aircraft remains the highest risk related to sUAS operations. The following priorities should be followed by operators during an in-flight emergency:
    
                    \paragraph{Aviate} When an aircraft is in the air, the top priority of the \ac{rpic} must always be to aviate. That means to maintain (or regain) manual control of the aircraft.
    
                    \paragraph{Navigate} After ensuring they have control of the aircraft, the \ac{rpic} must navigate the aircraft to the safest location possible in accordance with the mission parameters. \ac{rpic}'s must keep in mind that while this may be either \ac{lz}, or \ac{saa}, if the aircraft is unable to reach any of those locations safely, the \ac{rpic} must use their best judgement to land the aircraft in the safest manner possible considering the above priorities.
    
                    \paragraph{Communicate}  Once the \ac{rpic} has positive control of the aircraft they should communicate the issue appropriately as laid out in this \ac{sop}. This may be done simultaneous to the navigate step above, but should not take priority over safe navigation or operation of the aircraft.
            
    \section{Specific Emergency Procedures}

            \subsubsection{} The following specific \ac{ep}s reflect the most common emergencies encountered by \ac{suas} operators. These \ac{ep}s should be followed to the extent that they align with the priorities listed in \cref{sec:ep-general}.

        \subsection{Lost Communication (Lost Link)}\label{ssec:ep-losslink}
           
            \subsubsection{Situation}Lost Link refers to the loss of \ac{c2} between the \ac{gcs} and the \ac{suas} during flight. This includes primary and backup telemetry, command uplink, or video downlink loss beyond a pre-established threshold. Lost link may be temporary or permanent and can occur in both \ac{vlos} and \ac{bvlos} operations. All aircraft participating in \projectmars{} flights must be pre-programmed with an appropriate Lost Link procedure and contingency route.
        
            \subsubsection{Detection and Initial Assessment} Lost Link is detected when:
                
                \paragraph{} The \ac{gcs} indicates telemetry or control signal interruption (e.g., "No RC Signal," "Lost Telemetry").
                    
                \paragraph{} The aircraft ceases to respond to manual inputs.

                \paragraph{} A pre-set timeout value (e.g., 3 seconds for control link, 10 seconds for telemetry) is exceeded without recovery.
    
            \subsubsection*{}The Operator shall immediately verify the condition by confirming:\setcounter{paragraph}{0}
            
                \paragraph{}Power and antenna integrity at the \ac{gcs}.
                    
                \paragraph{}System status on secondary displays or via secondary control links (if available).
                    
                \paragraph{}Possible sources of local interference or terrain obstruction.
                    
            \subsubsection{Priorities}
        
                \paragraph{}Ensure safety of personnel, airspace users, and  infrastructure.

                \paragraph{}Prevent airspace incursion outside the authorized corridor.

                \paragraph{}Recover the aircraft safely to a pre-designated Rally Point or \ac{rth} location.

                \paragraph{}Maintain accountability of payloads.

                \paragraph{}Comply with \ac{faa}, \ac{dod}, and \ac{kach} safety policies.
                
            \subsubsection{Immediate Actions (All Flights)}

               \paragraph{}Attempt manual link recovery within 30 seconds.

                \paragraph{}If unsuccessful, monitor aircraft telemetry (if still available) and observe for programmed Lost Link behavior:
                    
                    \subparagraph{}Autonomous \ac{rth}
                
                    \subparagraph{}Loiter-and-Land
                    
                    \subparagraph{} Pre-programmed contingency route
                    
                \paragraph{} Notify XXX

                \paragraph{} If aircraft is non-responsive and on unsafe trajectory, initiate termination protocol.

                \paragraph{} Log time, location, telemetry status, and any deviation from planned route.
                                
            \subsubsection{\ac{bvlos} Specific Considerations}
      
            \subsubsection{\ac{vlos} Specific Considerations}
       
            \subsubsection{Reporting and Documentation}
       
            \subsubsection{Maintenance and Return to Service}
            
        \subsection{GPS/Navigation Failure}\label{ssec:ep-gpsfail}
        
            \subsubsection{Situation} \ac{gps} / Navigation Failure describes loss or severe degradation of onboard positioning data in primarily autonomous flight. When satellite signals, inertial sensors, or magnetometers become unreliable, the aircraft may revert to inertial dead reckoning, drift off the programmed track, or enter conservative failsafe behavior without human intervention. Operators must be prepared to monitor telemetry, switch to alternate navigation modes, and (if available) prompt the aircraft toward a safe loiter or landing using minimal manual inputs.

            \subsubsection{Detection and Initial Assessment}
            
            \subsubsection{Priorities}
            
            \subsubsection{Immediate Actions (All Flights)}
            
            \subsubsection{\ac{bvlos} Specific Considerations}
            
            \subsubsection{\ac{vlos} Specific Considerations}
            
            \subsubsection{Reporting and Documentation}
            
            \subsubsection{Maintenance and Return to Service}
        
        \subsection{Uncommanded Flight (Flyaway)}\label{ssec:ep-flyaway}
        
            \subsubsection{Situation}Uncommanded Flight (Flyaway) refers to autonomous aircraft motion that diverges from the programmed mission without human input. Control system anomalies, corrupt waypoints, sensor faults, or malicious interference can cause unexpected climbs, turns, or speed changes while automation remains engaged. Crew members must quickly recognize abnormal behavior, assess whether the autopilot remains responsive to high-level overrides, and be prepared to transition the platform to stabilized manual control or initiate termination protocols if containment fails.

            \subsubsection{Detection and Initial Assessment}
            
            \subsubsection{Priorities}
            
            \subsubsection{Immediate Actions (All Flights)}
            
            \subsubsection{\ac{bvlos} Specific Considerations}
            
            \subsubsection{\ac{vlos} Specific Considerations}
            
            \subsubsection{Reporting and Documentation}
            
            \subsubsection{Maintenance and Return to Service}
            
        \subsection{Intrusion of Local Airspace}\label{ssec:ep-intrusion}
        
            \subsubsection{Situation}Intrusion of Local Airspace occurs when an uncooperative aircraft enters the \ac{rwc} volume around an autonomously flown \ac{suas}. Unexpected manned or unmanned traffic may appear without prior coordination, penetrate the \ac{rwc} volume from any direction, and present an immediate midair collision risk while automation remains engaged. Operators must watch for \ac{daa} or geofence alerts, issue prompt reroute or hold commands to increase separation, and be prepared to assume manual control or execute termination protocols if collision risk cannot be mitigated.

            \subsubsection{Detection and Initial Assessment}
            
            \subsubsection{Priorities}
            
            \subsubsection{Immediate Actions (All Flights)}
            
            \subsubsection{\ac{bvlos} Specific Considerations}
            
            \subsubsection{\ac{vlos} Specific Considerations}
            
            \subsubsection{Reporting and Documentation}
            
            \subsubsection{Maintenance and Return to Service}
        
        \subsection{Low Battery Emergency Landing}\label{ssec:ep-lowbattery}
        
            \subsubsection{Situation}Low Battery Emergency Landing is triggered when an autonomously flown aircraft forecasts insufficient power to complete the programmed mission safely. Accelerated discharge, cell imbalance, or temperature-induced capacity loss may reduce available endurance without human inputs to conserve energy. The automation may command aggressive power-saving modes, altitude reductions, or immediate descent; operators should supervise these transitions, select the nearest safe landing area, and provide minimal manual guidance only as needed to ensure a controlled touchdown before critical levels are reached.

            \subsubsection{Detection and Initial Assessment}
            
            \subsubsection{Priorities}
            
            \subsubsection{Immediate Actions (All Flights)}
            
            \subsubsection{\ac{bvlos} Specific Considerations}
            
            \subsubsection{\ac{vlos} Specific Considerations}
            
            \subsubsection{Reporting and Documentation}
            
            \subsubsection{Maintenance and Return to Service}
            
    \section{Accident/Incident Reporting}\label{sec:accidentreporting}

        \subsubsection{} Standard operations of the \ac{uas} may involved flight termination within the operating area. Nominal operations may result in mid-air collisions between \projectmars{} \ac{uas} within the operating area, which may cause components to break. These events do not constitute mishaps that are reportable outside the \projectmars{} chain of command. Reportable mishaps include:

            \paragraph{}Flight outside approved airspace

            \paragraph{}\ac{suas} operations leading to injury

            \paragraph{}\ac{suas} operations leading to damge to property not invovled with \projectmars{}.
        
        \subsubsection{} In the event of a reportable mishap, crash, vehicle accident, or injury all \projectmars{} \ac{suas} are immediately grounded until cleared by the \namecref{role:projectoic} via \ac{mfr}.
        
        \subsubsection{\acf{rpic} Responsibilities}In the event of a \ac{uas} mishap, vehicle accident, or injury, the following steps will be taken by the \ac{rpic}:

            \paragraph{}Stop the mission.

            \paragraph{}In the event of injury to personnel, or ongoing threat to life or property (e.g. fire), dial 911.
            
            \paragraph{}Contact \namecref{role:projectncoic} about incident; if unable to reach \namecref{role:projectncoic}, escalate to \namecref{role:projectoic}.  If unable to reach either the \namecref{role:projectncoic} or \namecref{role:projectoic} and there are injuries, make contact with the \namecref{role:hospitalcommander}

            \paragraph{}Collect the following information:

                \subparagraph{}Type of \ac{uas}

                \subparagraph{}Model of \ac{uas}

                \subparagraph{}Type of Assistance Needed

                \subparagraph{}Location of incident

                \subparagraph{}Type and severity of injuries

                \subparagraph{}Names of injured

                \subparagraph{}Date and time

                \subparagraph{}Personnel and property involved in accident

                \subparagraph{}any additional hazards remaining (eg, fire/hazmat/etc)

            \paragraph{}Apply the following precautions:

                \subparagraph{}Keep others away for their own safety.

                \subparagraph{}Render first aid.

                \subparagraph{}Secure and control the accident site.

                \subparagraph{}Advise personnel help is on the way. 

                \subparagraph{}Remain at accident site until relieved either by \namecref{role:projectncoic} or \namecref{role:projectoic}.

        \subsubsection{\namecref{role:projectncoic} Responsibilities} Upon notification of a mishap, the \namecref{role:projectncoic} will:

            \paragraph{}Immediately move to the incident site to provide supervision.

            \paragraph{}Notify \namecref{role:projectoic} of incident. 

            \paragraph{}Ensure all aircraft and crewmember flight records are secured. 

            \paragraph{}Recover the aircraft when safe.

            \paragraph{}Be prepared to brief the \namecref{role:hospitalcommander} as requested.

            \paragraph{}Provide resources and assistance to accident investigators as necessary.

        \subsubsection{\namecref{role:projectoic} Responsibilities} Upon notification of a mishap, the \namecref{role:projectoic} will:

            \paragraph{}Notify the \namecref{role:hospitalcommander} as appropriate.

            \paragraph{}Identify the nature of the mishap and appoint appropriate personnel to investigate. In the event of injury, damage to non-Project material, and in other circumstances the \ac{oic} deems appropriate, the \namecref{role:projectoic} will recommend appointment of \ac{io} to the chain of command.

            \paragraph{}Review findings and recommendations of \ac{io}, if appointed, or \namecref{role:projectncoic} and record course of action and approval for resumption of flight operations via \ac{mfr}.