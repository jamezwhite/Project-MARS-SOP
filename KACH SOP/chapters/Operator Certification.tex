\chapter{Pilot Certification, Training, \& Currency}
\label{ch:certification}

        \subsubsection{}This chapter establishes the \acf{atp} for \projectmars{}.  \projectmars{}' primarily operates nonstandard, \ac{uas}s with a non-tactical mission, and its qualification, evaluation, and currency requirements are defined in AR 95-1, Appendix D, Paragraph 10.  The \projectmars{} \ac{atp} provides specific guidelines for executing \ac{suas} aircrew training, and establishes crewmember qualification, refresher, mission, and continuation training and evaluation requirements.

        \subsubsection{} The operator’s manual is the governing authority for operation of the aircraft. If differences exist between the maneuver descriptions in the operators’ manual and this chapter, then this chapter the governing authority for training and flight evaluation purposes only. 

    \section{Operator Qualification Framework}\label{sec:qual-framework}

            \subsubsection{} The \projectmars{} operator qualification framework is a tiered system designed to ensure personnel possess the appropriate skills and knowledge for the missions they are assigned. Advancement through these tiers is based on a structured progression of training, demonstrated proficiency, and a comprehensive understanding of program procedures. This framework ensures that while all operators meet a baseline safety standard, those tasked with more complex or higher-risk missions have achieved a corresponding level of expertise.

            \subsubsection{} The three levels of qualification are \acf{ao} , \acf{qp} , and \acf{eo}. 

                \paragraph{\acl{ao}} An \acl{ao} represents the entry-level qualification, trained specifically to oversee and manage pre-planned, routine autonomous missions under normal conditions. 

               \paragraph{\acl{qp}}The next level, \acl{qp} , builds upon that foundation by adding comprehensive manual flight proficiency. A \ac{qp} is skilled in taking direct manual control of the aircraft, especially during off-nominal situations, confined area landings, or when deviating from an autonomous plan. 

                \paragraph{\acl{eo}}The highest level of qualification is the \acl{eo} , an individual with extensive experience who serves as an instructor, evaluator, and subject matter expert. \ac{eo}s are responsible for conducting training, certifying other operators, and leading the development and validation flights for new routes and emergency procedures.

    \section{Requirements for All Operators}\label{sec:qual-reqsall}
    
            \subsubsection{} The following personnel may fly and/or operate \projectmars{} airframes.  Operators who --

                    \paragraph{} Are members of the Regular Army, \ac{usar}, \ac{arng}, or Civilian employees of the U.S. Army or \ac{dha}.

                    \paragraph{} Have complied with the qualification, training, evaluation, and currency requirements of this SOP for the airframe to be flown and/or operated.

                    \paragraph{} Meet the medical standard as outlined in AR 40-501 (but are not required to maintain a class IV physical).

    \section{\acf{ao} Qualification}\label{sec:qual-ao}
    
            \subsubsection{Requirements} \acf{ao} level is the most basic operator certification level, and consists of \ac{buq} 1 and 2 certification completed via the SUASMAN website (or equivalent), plus 2 hours local training on the \projectmars{} airframe.  
            
            \subsubsection{Role}This level qualifies the \ac{ao} to operate \projectmars{} \ac{suas}s in autonomous mode  only. \ac{ao} qualified personnel will not conduct operations without the assignment of a supporting \ac{qp} or \ac{eo}. 
        
            \subsubsection{Re-certification}In addition to the currency requirements listed below, Personnel trained to \ac{ao} Level are required to re-certify with a currently qualified \ac{eo} annually. 

    \section{\acf{qp} Qualification}\label{sec:qual-qp} 
            
            \subsubsection{Requirements}\acf{qp} level is the intermediate level of qualification for \projectmars{}. In addition to the requirements of \nameref{sec:qual-ao}, \ac{qp} qualification requires graduation from the \ac{suas} Operator Course.

            \subsubsection{Role}A \ac{qp} is skilled in taking direct manual control of the aircraft, especially during off-nominal situations, confined area landings, or when deviating from an autonomous plan. In addition to autonomous operations, the \ac{qp} is authorized to conduct flights in manual flight modes, such a \ac{dte} and training flights.

            \subsubsection{Re-certification}In addition to the currency requirements listed below, Personnel trained to \ac{qp} Level are required to re-certify with the \nameref{role:leadoperator} annually.

            
    \section{\acf{eo} Qualification}\label{sec:qual-eo}

            \subsubsection{Requirements}\acf{eo} qualification is the highest level of qualification recognized by \projectmars{}.  In addition to the requirements for \nameref{sec:qual-qp}, \ac{eo} qualification requires Appointment Orders signed by the \nameref{role:projectoic}.  While not a qualification requirement, attendance of the \ac{suas} Master Trainer Course is strongly encouraged for \ac{eo}, and they will be prioritized for attendance. 

            \subsubsection{Role} As the highest trained operators within \projectmars{}, \ac{eo}s will act as \ac{sme}s for \ac{suas} operations as well as flight instructors and operator evaluators for all \projectmars{} personnel. The \nameref{role:leadoperator} must be \ac{eo} qualified.

            \subsubsection{Re-certification}In addition to the currency requirements listed below, Personnel trained to \ac{eo} Level are required to re-certify with the \nameref{role:leadoperator} annually.
     
    \section{Currency and Recurrency Requirements}\label{sec:currencyreqs}

            \subsubsection{Currency} Currency requirements for \projectmars{} operators are dependent on the operator's level of qualification. 

                \paragraph{Frequency} To be considered current, a \projectmars{} \ac{suas} operator must:

                    \subparagraph{\acf{ao}} Every 60 consecutive days, perform one flight each to two different locations under autonomous conditions.  Receive at least one autonomous mission launched from another \ac{lz}. 

                    \subparagraph{\acf{qp}} Every 30 consecutive days, perform a launch, recovery, and 15 minute piloted (i.e., non-autonomous) flight of a \projectmars{} \ac{suas} (or compatible simulator). \ac{qp}s will perform a live launch, recovery, and a 15-minute piloted flight of the \ac{suas} every 150 consecutive days.

                    \subparagraph{\acf{eo}} Currency requirements for \ac{eo} are the same as those for \ac{qp}s.

                \paragraph{Tracking} Tracking actual flight time for a flying hour requirement is impractical and not required. Individual flight records folders are not required; however, documentation of flight operations (sorites) for the purpose of tracking currency is required. A qualified sortie is a launch, recovery, and autonomous flight for \ac{ao} currency, and a launch, recovery, and piloted flight for \ac{qp} or \ac{eo} currency.

                \paragraph{Currency Lapse} \projectmars{} operators whose currency has lapsed must complete a proficiency flight evaluation by the \nameref{role:leadoperator} before conducting operations. Simulators may not be used to reestablish currency.

                \paragraph{Waivers} Waivers to currency may only be granted by the \nameref{role:projectoic}.

                \paragraph{Similar \ac{uas}}  Currency in one \projectmars{} \ac{suas} will satisfy requirement for all \ac{suas} belonging to the \projectmars{}.  Autonomous flights do not satisfy requirement for piloted flights. 
                