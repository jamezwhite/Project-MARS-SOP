\chapter{Pilot Certification, Training, \& Currency}
\label{ch:certification}

        \subsubsection{}This chapter establishes the \acf{atp} for Project \ac{mars}.  Project \ac{mars}' primarily operates nonstandard, \ac{uas}s with a non-tactical mission, and its qualification, evaluation, and currency requirements are defined in AR 95-1, Appendix D, Paragraph 10.  The Project \ac{mars} \ac{atp} provides specific guidelines for executing \ac{suas} aircrew training, and establishes crewmember qualification, refresher, mission, and continuation training and evaluation requirements.

        \subsubsection{} The operator’s manual is the governing authority for operation of the aircraft. If differences exist between the maneuver descriptions in the operators’ manual and this chapter, then this chapter the governing authority for training and flight evaluation purposes only. 

    \section{Operator Qualification Framework}

            \subsubsection{} The Project \ac{mars} operator qualification framework is a tiered system designed to ensure personnel possess the appropriate skills and knowledge for the missions they are assigned. Advancement through these tiers is based on a structured progression of training, demonstrated proficiency, and a comprehensive understanding of program procedures. This framework ensures that while all operators meet a baseline safety standard, those tasked with more complex or higher-risk missions have achieved a corresponding level of expertise.

            \subsubsection{} The three levels of qualification are \acf{ao} , \acf{qp} , and \acf{eo} . 

                \paragraph{\acl{ao}} An \acl{ao} represents the entry-level qualification, trained specifically to oversee and manage pre-planned, routine autonomous missions under normal conditions. 
\\
               \paragraph{\acl{qp}}The next level, \acl{qp} , builds upon that foundation by adding comprehensive manual flight proficiency. A \ac{qp} is skilled in taking direct manual control of the aircraft, especially during off-nominal situations, confined area landings, or when deviating from an autonomous plan. 
\\
                \paragraph{\acl{eo}}The highest level of qualification is the \acl{eo} , an individual with extensive experience who serves as an instructor, evaluator, and subject matter expert. \ac{eo}s are responsible for conducting training, certifying other operators, and leading the development and validation flights for new routes and emergency procedures.

    \section{Requirements for All Operators}
    
            \subsubsection{} The following personnel may fly and/or operate Project \ac{mars} airframes.  Operators who --
\\
                    \paragraph{} Are members of the Regular Army, \ac{usar}, \ac{arng}, or Civilian employees of the U.S. Army or \ac{dha}.
\\
                    \paragraph{} Have complied with the qualification, training, evaluation, and currency requirements of this SOP for the airframe to be flown and/or operated.
\\
                    \paragraph{} Meet the medical standard as outlined in AR 40-501 (but are not required to maintain a class IV physical).
\\
    \section{\acf{ao} Qualification}
    
            \subsubsection{} \acf{ao} level is the most basic operator certification level, and consists of \ac{buq} 1 and 2 certification completed via the SUASMAN website (or equivalent), plus 2 hours local training on the Project \ac{mars} airframe.  This level certifies the pilot to operate the \ac{kach} airframe in autonomous mode  only, under supervision of a higher-level Pilot Evaluator.
        
            \subsubsection{}Personnel trained to \ac{ao} Level are required to recertify with a currently qualified Pilot Evaluator annually. 

    \section{\acf{qp} Qualification}
   
    \section{\acf{eo} Qualification}
  
    \section{Instructor Pilot (IP) Training / Pilot Evaluator Training}
   
    \section{Currency and Recurrency Requirements}