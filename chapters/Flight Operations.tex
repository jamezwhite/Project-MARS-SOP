\chapter{Flight Operations}
\label{ch:flightops}

\claudeadd{This chapter establishes flight operations procedures for \projectmars{} \ac{suas} operations conducted from \ac{kach} across the \ac{wp} Garrison. Operations are conducted under both \ac{faa} Part 107 with \ac{coa} and \ac{dod}/Army regulations. This chapter addresses both \ac{vlos} operations (used during development, testing, and training) and \ac{bvlos} operations (used during routine autonomous delivery missions).}

\section{Ground Operations}\label{sec:groundops}

Ground operations encompass all activities required to prepare, launch, recover, and secure \projectmars{} \ac{suas} and associated equipment.

    \subsection{Launch and Landing Zone Setup}\label{ssec:lzsetup}

        \subsubsection{Primary Launch Site} The primary launch site for \projectmars{} operations is \ac{kach}. The designated \ac{lz} will be established in accordance with the approved \ac{coa} and coordinated with hospital operations. The \ac{lz} will be:

        \begin{itemize}
            \item Clear of obstacles within a 15-meter radius of the launch/landing point
            \item Marked or cordoned to prevent unauthorized personnel entry during operations
            \item Positioned to provide adequate clearance for approach and departure routes
            \item Accessible for \ac{gcs} setup with clear line of sight (for \ac{vlos} operations)
        \end{itemize}

        \subsubsection{Destination Landing Zones} Each destination \ac{lz} will be surveyed and approved prior to first use. Destination \acp{lz} serving \ac{usma} and \ac{usmaps} will be identified during route development and documented in mission planning materials. Each destination \ac{lz} will have:

        \begin{itemize}
            \item A designated point of contact for receiving operations
            \item Clearly defined boundaries and approach corridors
            \item Emergency contact information posted
        \end{itemize}

    \subsection{Ground Control Station Setup}\label{ssec:gcssetup}

        \subsubsection{} \claudeadd{The \ac{gcs} will be positioned to provide optimal telemetry reception and, for \ac{vlos} operations, direct line of sight to the \ac{lz} and initial flight path. The \nameref{role:rpic} will verify:}

        \begin{itemize}
            \item \claudeadd{\ac{gcs} is powered and all telemetry links are established}
            \item \claudeadd{Antenna orientation is correct for the planned flight path}
            \item \claudeadd{All software applications are running and displaying current data}
            \item \claudeadd{Backup power is available for extended operations}
        \end{itemize}

    \subsection{Payload Handling}\label{ssec:payloadhandling}

        \subsubsection{} Medical supply payloads will be prepared and loaded in accordance with \cref{sec:approvedcargo} (/nameref{sec:approvedcargo}). Personnel handling payloads will:

        \begin{itemize}
            \item Verify payload weight does not exceed aircraft limits
            \item Ensure payload is properly secured and will not shift during flight
            \item Confirm payload does not interfere with aircraft sensors or control surfaces
            \item Document payload serial number and weight in the flight log
        \end{itemize}

    \subsection{Site Security}\label{ssec:sitesecurity}

        \subsubsection{} During flight operations, the \ac{lz} and \ac{gcs} location will be secured to prevent unauthorized access. At minimum, one crewmember will maintain situational awareness of the immediate area to identify and respond to ground hazards. For \ac{vlos} operations, \nameref{role:visualobserver}s will be positioned to maintain both aircraft observation and ground security awareness.

\section{Flight Scheduling}\label{sec:flightscheduling}

\claudeadd{All \projectmars{} flights require coordination with multiple agencies and must be scheduled in advance through the established process.}

    \subsection{Weekly Scheduling Process}\label{ssec:weeklyschedule}

        \subsubsection{} The \nameref{role:projectncoic} will develop a weekly flight schedule and present it at the weekly leadership meeting for approval by the \nameref{role:projectoic}. The schedule will:

        \begin{itemize}
            \item Draft a preliminary schedule for two weeks out  
            \item Set the flight schedule for the following week
            \item Identify required crew assignments and equipment for the following week
            \item Note any special coordination requirements
        \end{itemize}

    \subsection{Required Coordination}\label{ssec:requiredcoord}

        \subsubsection{} \claudeadd{Prior to flight operations, the following coordination will be completed:}

            \paragraph{2nd Aviation Detachment} \claudeadd{The \nameref{role:projectncoic} will coordinate with 2nd Aviation Detachment for airspace deconfliction no later than three hours prior to flight. 2nd Aviation will be notified within 15 minutes of flight completion.}

            \paragraph{\ac{usma} \ac{g3}} \claudeadd{The \nameref{role:projectncoic} will consult with \ac{usma} \ac{g3} during weekly mission planning to ensure flight operations do not conflict with installation activities.}

            \paragraph{\ac{faa} \ac{notam}} A \ac{notam} will be published for all flight operations in accordance with \ac{coa} requirements; no later than 24 hours prior to flight. The \nameref{role:rpic} will verify \ac{notam} publication prior to flight.

            \paragraph{\ac{wp} \ac{des}} The \ac{wp} \ac{des} Duty Desk (845-938-3333) will be notified no later than 10 minutes prior to flight operations and within 15 minutes of flight completion.

\section{Briefings and Risk Management}\label{sec:Briefings}

\claudeadd{A structured briefing process ensures all personnel understand mission requirements, hazards, and contingency procedures.}

    \subsection{Weekly Leadership Brief}\label{ssec:weeklyleadershipbrief}

        \subsubsection{} The \nameref{role:projectncoic} will conduct a weekly leadership briefing attended by the \nameref{role:projectoic}, \nameref{role:leadoperator}, and other stakeholders as appropriate. The briefing will cover:

        \begin{itemize}
            \item Review of previous week's operations and any lessons learned
            \item Approval of the upcoming week's flight schedule
            \item Discussion of any new hazards or risk mitigations
            \item Equipment status and maintenance issues
            \item Training and qualification updates
        \end{itemize}

    \subsection{Mission Brief}\label{ssec:missionbrief}

        \subsubsection{} \claudeadd{A mission brief will be conducted the day before scheduled flights. For \ac{vlos} developmental flights, the brief will be conducted by the \nameref{role:leadoperator} or designated \ac{rpic}. The mission brief will include:}

        \begin{itemize}
            \item \claudeadd{Mission objectives and planned routes}
            \item \claudeadd{Weather forecast and go/no-go criteria}
            \item \claudeadd{Crew assignments and responsibilities}
            \item \claudeadd{Aircraft and equipment assignments}
            \item \claudeadd{Communication procedures and frequencies}
            \item \claudeadd{Emergency procedures and \acp{saa}}
            \item \claudeadd{Known hazards and mitigations}
        \end{itemize}

    \subsection{Crew Brief}\label{ssec:crewbrief}

        \subsubsection{} \claudeadd{On the day of flight, the \ac{rpic} will conduct a crew brief immediately prior to operations. The crew brief will confirm:}

        \begin{itemize}
            \item \claudeadd{Current weather conditions meet minimums}
            \item \claudeadd{All crewmembers understand their roles}
            \item \claudeadd{Emergency procedures are understood}
            \item \claudeadd{Communication checks are complete}
            \item \claudeadd{Any changes from the mission brief}
        \end{itemize}

    \subsection{Composite Risk Management}\label{ssec:crm}

        \subsubsection{} \claudeadd{\ac{crm} will be conducted in accordance with DA Pamphlet 385-30 using DA Form 7566. The \nameref{role:projectncoic} will ensure a current \ac{crm} assessment exists for each flight profile. The \ac{crm} will:}

        \begin{itemize}
            \item \claudeadd{Identify hazards specific to the planned operation}
            \item \claudeadd{Assess initial risk levels}
            \item \claudeadd{Develop controls and mitigations}
            \item \claudeadd{Determine residual risk}
            \item \claudeadd{Obtain appropriate risk acceptance authority signature}
        \end{itemize}

        \subsubsection{} \claudeadd{The \ac{raa} for \projectmars{} operations is established in \cref{ch:safety}. The \ac{rpic} will review the applicable \ac{crm} assessment prior to each flight and verify all controls remain in place.}

\section{Weather Considerations}\label{sec:weather}

\claudeadd{\projectmars{} operations will comply with \ac{faa} Part 107 weather minimums unless more restrictive limits are specified in the applicable \ac{coa} or this \ac{sop}.}

    \subsection{Weather Minimums}\label{ssec:weathermins}

        \subsubsection{} \claudeadd{The following weather minimums apply to all \projectmars{} flight operations:}

        \begin{itemize}
            \item \claudeadd{Visibility: 3 statute miles minimum}
            \item \claudeadd{Cloud clearance: 500 feet below clouds, 2000 feet horizontal from clouds}
            \item \claudeadd{Maximum altitude: 400 feet \ac{agl}}
            \item \claudeadd{Daylight operations only (civil twilight to civil twilight)}
        \end{itemize}

        \subsubsection{} \claudeadd{Operations during civil twilight require the \ac{suas} to be equipped with anti-collision lights visible for at least 3 statute miles.}

    \subsection{Go/No-Go Criteria}\label{ssec:gonogo}

        \subsubsection{} \claudeadd{In addition to weather minimums, the following conditions constitute automatic no-go criteria:}

        \begin{itemize}
            \item \claudeadd{Sustained winds exceeding aircraft limitations}
            \item \claudeadd{Gusts exceeding aircraft limitations}
            \item \claudeadd{Precipitation (rain, snow, sleet)}
            \item \claudeadd{Lightning within 10 nautical miles}
            \item \claudeadd{Icing conditions}
            \item \claudeadd{\ac{wp} \ac{fpcon} at CHARLIE or DELTA}
        \end{itemize}

    \subsection{Weather Decision Authority}\label{ssec:weatherdecision}

        \subsubsection{} \claudeadd{Weather decisions follow a tiered approval chain:}

            \paragraph{} \claudeadd{The \ac{rpic} is responsible for evaluating current conditions and making the initial weather determination.}

            \paragraph{} \claudeadd{The \nameref{role:projectncoic} will approve weather-related go/no-go decisions for routine operations.}

            \paragraph{} \claudeadd{The \nameref{role:projectoic} will provide final sign-off on weather decisions that involve marginal conditions or deviation from standard procedures.}

    \subsection{Weather Hold Procedures}\label{ssec:weatherhold}

        \subsubsection{} \claudeadd{If weather conditions deteriorate during operations, the \ac{rpic} will:}

        \begin{itemize}
            \item \claudeadd{Immediately assess whether to continue, hold, or abort the mission}
            \item \claudeadd{For deteriorating but still acceptable conditions: continue with increased monitoring}
            \item \claudeadd{For marginal conditions: place the mission on weather hold and await improvement}
            \item \claudeadd{For conditions below minimums or approaching rapidly: execute immediate recovery or abort}
        \end{itemize}

        \subsubsection{} \claudeadd{Emergency weather procedures are detailed in \cref{ssec:ep-weather}.}

\section{Crew Coordination}\label{sec:crewcoordination}

\claudeadd{Effective crew coordination is essential for safe flight operations. Crew composition and responsibilities differ between \ac{vlos} developmental operations and \ac{bvlos} autonomous operations.}

    \subsection{VLOS Operations Crew}\label{ssec:vloscrew}

        \subsubsection{} \claudeadd{\ac{vlos} operations typically involve 3-5 personnel for route assessment, development, and validation flights:}

            \paragraph{\nameref{role:rpic}} \claudeadd{Responsible for overall flight safety and aircraft control. Maintains authority over all flight decisions.}

            \paragraph{\nameref{role:visualobserver}(s)} \claudeadd{Maintain visual contact with the aircraft and provide situational awareness to the \ac{rpic}. Multiple \nameref{role:visualobserver}s may be positioned along the route to maintain continuous visual contact.}

            \paragraph{Ground Observers} \claudeadd{Additional personnel positioned along the route to observe aircraft performance, identify hazards, and validate route suitability. Ground observers will have direct communication with the \ac{rpic}.}

    \subsection{BVLOS Operations Crew}\label{ssec:bvloscrew}

        \subsubsection{} \claudeadd{\ac{bvlos} autonomous delivery operations require the following minimum crew:}

            \paragraph{Launch \ac{rpic}} \claudeadd{Located at \ac{kach}, responsible for pre-flight, launch, and monitoring during transit. Maintains authority to abort the mission.}

            \paragraph{Receiving Pilot} \claudeadd{Located at the destination \ac{lz}, responsible for receiving the aircraft and confirming safe landing. The receiving pilot will be at minimum \ac{ao} qualified and in direct communication with the launch \ac{rpic}.}

    \subsection{Communication Protocols}\label{ssec:commprotocols}

        \subsubsection{} \claudeadd{All crewmembers will maintain communication throughout flight operations using approved methods (radio, telephone, or other means as specified in the mission brief). Standard communication calls include:}

        \begin{itemize}
            \item \claudeadd{``Ready for launch'' -- \ac{rpic} confirms all pre-flight complete}
            \item \claudeadd{``Launching'' -- Aircraft is taking off}
            \item \claudeadd{``Aircraft in sight'' -- \nameref{role:visualobserver}/Ground Observer confirms visual contact}
            \item \claudeadd{``Lost visual'' -- \nameref{role:visualobserver} has lost sight of aircraft}
            \item \claudeadd{``Approaching [waypoint]'' -- Position report during autonomous flight}
            \item \claudeadd{``On final'' -- Aircraft on approach to \ac{lz}}
            \item \claudeadd{``Touchdown'' -- Aircraft has landed}
            \item \claudeadd{``Secured'' -- Aircraft is powered down and safe}
        \end{itemize}

    \subsection{Handoff Procedures}\label{ssec:handoff}

        \subsubsection{} \claudeadd{For \ac{bvlos} operations, control authority handoff between launch \ac{rpic} and receiving pilot will be conducted as follows:}

        \begin{itemize}
            \item \claudeadd{Launch \ac{rpic} maintains authority throughout transit}
            \item \claudeadd{When aircraft enters visual range of destination, receiving pilot confirms ``Aircraft in sight''}
            \item \claudeadd{Launch \ac{rpic} may transfer landing authority to receiving pilot if situation requires}
            \item \claudeadd{Transfer of authority will be verbally confirmed by both pilots}
            \item \claudeadd{After landing, receiving pilot assumes responsibility for aircraft security}
        \end{itemize}

\section{Types of Operations}\label{sec:opstypes}

\claudeadd{\projectmars{} conducts several types of flight operations, each with distinct procedures and crew requirements.}

    \subsection{Routine Scheduled Delivery}\label{ssec:routinedelivery}

        \subsubsection{} \claudeadd{Routine scheduled delivery missions are \ac{bvlos} autonomous flights conducted on pre-validated routes. These missions:}

        \begin{itemize}
            \item \claudeadd{Follow routes that have been previously surveyed and approved during developmental testing}
            \item \claudeadd{Are scheduled through the weekly planning process}
            \item \claudeadd{Require minimum crew of launch \ac{rpic} and receiving pilot}
            \item \claudeadd{Follow pre-programmed flight paths with minimal deviation}
        \end{itemize}

        \subsubsection{Procedures}

        \begin{itemize}
            \item \claudeadd{Mission brief completed day prior}
            \item \claudeadd{Crew brief conducted morning of flight}
            \item \claudeadd{Pre-flight inspection and payload loading per \cref{ssec:preflight}}
            \item \claudeadd{Launch per approved procedures}
            \item \claudeadd{\ac{rpic} monitors telemetry throughout transit}
            \item \claudeadd{Receiving pilot confirms approach and landing}
            \item \claudeadd{Post-flight documentation completed}
        \end{itemize}

    \subsection{On-Demand / Emergency Response Delivery}\label{ssec:ondemanddelivery}

        \subsubsection{} \claudeadd{On-demand and emergency response delivery operations are not within the current scope of \projectmars{}. Procedures for these operations will be developed as the program matures and expands beyond the developmental phase.}

    \subsection{Development, Testing, and Evaluation Flights}\label{ssec:testflights}

        \subsubsection{} \claudeadd{\ac{dte} flights are conducted under \ac{vlos} conditions to survey routes, validate aircraft performance, and refine operational procedures. These flights:}

        \begin{itemize}
            \item \claudeadd{Are the primary flight type during the developmental phase}
            \item \claudeadd{Require full \ac{vlos} crew (3-5 personnel)}
            \item \claudeadd{Allow the \ac{rpic} flexibility to adapt within a few dozen meters of the planned route}
            \item \claudeadd{Generate data used to approve routes for future \ac{bvlos} operations}
        \end{itemize}

        \subsubsection{Procedures}

        \begin{itemize}
            \item \claudeadd{Route survey and hazard identification}
            \item \claudeadd{\ac{crm} assessment specific to test objectives}
            \item \claudeadd{Positioning of \nameref{role:visualobserver}s and ground observers along route}
            \item \claudeadd{Incremental testing (low altitude, then full profile)}
            \item \claudeadd{Documentation of route performance and any required modifications}
            \item \claudeadd{Post-flight debrief and data collection}
        \end{itemize}

    \subsection{Training Flights}\label{ssec:trainingflights}

        \subsubsection{} \claudeadd{Training flights are conducted to qualify and maintain currency for \projectmars{} operators. Training flights require increased supervision and are restricted to approved training areas.}

        \subsubsection{} \claudeadd{All training flights will:}

        \begin{itemize}
            \item \claudeadd{Be supervised by a \ac{qp} or higher qualified operator}
            \item \claudeadd{Be conducted in designated training areas with minimal risk to non-participants}
            \item \claudeadd{Follow the training requirements outlined in \cref{ch:certification}}
            \item \claudeadd{Be documented in the trainee's qualification record}
        \end{itemize}

        \subsubsection{} \claudeadd{Risk management for training flights is addressed in \cref{ch:safety}.}

\section{Air Corridor and Route Management}\label{sec:aircorridormgmt}

\claudeadd{Routes and air corridors for \projectmars{} operations are developed, validated, and managed through a structured process to ensure safe and repeatable operations.}

    \subsection{Route Development}\label{ssec:routedevelopment}

        \subsubsection{} \claudeadd{New routes will be developed using the following process:}

        \begin{itemize}
            \item \claudeadd{Ground survey of proposed route to identify obstacles, hazards, and suitable \acp{lz}}
            \item \claudeadd{Identification of primary, alternate, and contingency routes for each destination}
            \item \claudeadd{Designation of \acp{saa} along each route}
            \item \claudeadd{\ac{vlos} test flights to validate route suitability}
            \item \claudeadd{Documentation and approval of route for operational use}
        \end{itemize}

    \subsection{Primary, Alternate, and Contingency Routes}\label{ssec:routeplanning}

        \subsubsection{} \claudeadd{For each destination, the following routes will be established:}

            \paragraph{Primary Route} \claudeadd{The preferred flight path under normal conditions. Selected for optimal efficiency while maintaining safety margins.}

            \paragraph{Alternate Route} \claudeadd{An alternative flight path available when the primary route is unavailable (e.g., due to conflicting activities or temporary obstacles).}

            \paragraph{Contingency Route} \claudeadd{A simplified route for use during degraded conditions or emergencies. Contingency routes prioritize safety and may include additional \acp{saa}.}

        \subsubsection{} \claudeadd{Routes will be identified in the weekly mission brief and loaded into the autonomous flight plan prior to each mission.}

    \subsection{Safe Abort Areas}\label{ssec:safeabortareas}

        \subsubsection{} \claudeadd{\acp{saa} are pre-surveyed locations along each route where the aircraft can safely land in an emergency. Each route will have multiple \acp{saa} spaced to ensure the aircraft can always reach one if required.}

        \subsubsection{} \claudeadd{\ac{saa} locations will be:}

        \begin{itemize}
            \item \claudeadd{Programmed as rally points in the autonomous flight plan}
            \item \claudeadd{Clear of obstacles and away from populated areas}
            \item \claudeadd{Accessible for aircraft recovery}
            \item \claudeadd{Documented with coordinates and description}
        \end{itemize}

        \subsubsection{} \claudeadd{In the event of an anomaly requiring abort, the aircraft will navigate to the nearest \ac{saa} and land. Lost link procedures are configured to direct the aircraft to the nearest \ac{saa}.}

    \subsection{Geofencing and Corridor Boundaries}\label{ssec:geofencing}

        \subsubsection{} \claudeadd{All \ac{bvlos} flights will operate within defined corridor boundaries enforced by geofencing. The geofence will:}

        \begin{itemize}
            \item \claudeadd{Define lateral boundaries of the approved flight corridor}
            \item \claudeadd{Enforce altitude limits in accordance with the \ac{coa}}
            \item \claudeadd{Trigger alerts if the aircraft approaches boundaries}
            \item \claudeadd{Initiate automatic abort procedures if boundaries are breached}
        \end{itemize}

        \subsubsection{} \claudeadd{For \ac{vlos} operations, the \ac{rpic} may adjust within a few dozen meters of the planned route as needed for safety or mission accomplishment, but will remain within the overall approved operating area.}

\section{Mission Planning and Approval Cycle}\label{sec:missionplanningcycle}

\claudeadd{All \projectmars{} flight operations require planning and approval through the established cycle. Different types of operations require different levels of approval.}

    \subsection{Approval Authority Matrix}\label{ssec:approvalmatrix}

        \subsubsection{} \claudeadd{The following table summarizes approval authorities for \projectmars{} operations:}

        \begin{center}
        \begin{tabular}{|l|l|l|}
        \hline
        \textbf{Decision} & \textbf{Approval Authority} & \textbf{Consulted/Informed} \\
        \hline
        \claudeadd{Overall Operational Risk} & \claudeadd{\nameref{role:hospitalcommander}} & \claudeadd{--} \\
        \hline
        \claudeadd{Airspace Authorization} & \claudeadd{\ac{faa} via \ac{usaasa}} & \claudeadd{--} \\
        \hline
        \claudeadd{Ground Operations Authority} & \claudeadd{\ac{wp} Garrison Commander} & \claudeadd{\ac{smc}} \\
        \hline
        \claudeadd{Weekly Flight Schedule} & \claudeadd{\nameref{role:projectoic}} & \claudeadd{\ac{usma} \ac{g3}, 2nd AVN} \\
        \hline
        \claudeadd{Weather Go/No-Go} & \claudeadd{\nameref{role:projectoic}} & \claudeadd{\nameref{role:projectncoic}} \\
        \hline
        \claudeadd{Equipment Airworthiness} & \claudeadd{\ac{rpic}} & \claudeadd{\nameref{role:technicallead}} \\
        \hline
        \claudeadd{Final Launch Authority} & \claudeadd{\ac{rpic}} & \claudeadd{Flight crew, \ac{wp} \ac{des}} \\
        \hline
        \end{tabular}
        \end{center}

    \subsection{Weekly Planning Cycle}\label{ssec:weeklyplanning}

        \subsubsection{} \claudeadd{The weekly planning cycle proceeds as follows:}

        \begin{itemize}
            \item \claudeadd{\textbf{Week -2:} \nameref{role:projectncoic} drafts preliminary schedule}
            \item \claudeadd{\textbf{Week -1:} Weekly leadership meeting reviews and approves schedule; \nameref{role:projectncoic} coordinates with \ac{usma} \ac{g3} and 2nd AVN}
            \item \claudeadd{\textbf{Week -1:} \ac{notam}s submitted for publication}
            \item \claudeadd{\textbf{Day -1:} Mission brief conducted}
            \item \claudeadd{\textbf{Day 0:} Crew brief, pre-flight, and flight execution}
        \end{itemize}

    \subsection{Mission-Specific Approvals}\label{ssec:missionapprovals}

        \subsubsection{} \claudeadd{Certain mission types require additional approvals:}

        \begin{itemize}
            \item \claudeadd{Flights over non-pre-established routes require \ac{usma} \ac{g3} coordination at least 5 working days prior}
            \item \claudeadd{Night operations or operations over people require \ac{smc} approval and additional \ac{faa} waivers}
            \item \claudeadd{Flights in areas requiring Range Facility Management System (RFMIS) reservation require \ac{dptms} approval}
        \end{itemize}

\section{Flight Procedures}\label{sec:flightprocedures}

\claudeadd{This section establishes standardized procedures for all phases of flight. Personnel will follow these procedures unless deviation is required for safety of flight.}

    \subsection{Pre-Flight Procedures}\label{ssec:preflight}

        \subsubsection{} \claudeadd{The following pre-flight procedures will be completed before every flight:}

            \paragraph{Documentation Review} \claudeadd{The \ac{rpic} will verify:}
            \begin{itemize}
                \item \claudeadd{Current \ac{awr} on file for the aircraft}
                \item \claudeadd{\ac{notam} is published and active}
                \item \claudeadd{No \acp{tfr} or other airspace restrictions affect the flight}
                \item \claudeadd{\ac{crm} assessment is current and controls are in place}
            \end{itemize}

            \paragraph{Coordination Notifications} \claudeadd{The \ac{rpic} or designated crewmember will:}
            \begin{itemize}
                \item \claudeadd{Contact 2nd AVN for airspace deconfliction (NLT 3 hours prior)}
                \item \claudeadd{Notify \ac{wp} \ac{des} (NLT 10 minutes prior)}
                \item \claudeadd{Confirm receiving crewmember is in position (if \ac{bvlos})}
            \end{itemize}

            \paragraph{Aircraft Inspection} \claudeadd{The \ac{rpic} will inspect:}
            \begin{itemize}
                \item \claudeadd{Airframe integrity (no visible damage, all components secure)}
                \item \claudeadd{Propellers (no nicks, cracks, or damage)}
                \item \claudeadd{Battery charge and condition}
                \item \claudeadd{Control surfaces and motors (free movement, secure mounting)}
                \item \claudeadd{Sensors and cameras (clean and unobstructed)}
                \item \claudeadd{Payload attachment and security}
            \end{itemize}

            \paragraph{\ac{gcs} Verification} \claudeadd{The \ac{rpic} will verify:}
            \begin{itemize}
                \item \claudeadd{Telemetry link established}
                \item \claudeadd{\ac{gps} signal acquired (sufficient satellites)}
                \item \claudeadd{Flight plan loaded correctly}
                \item \claudeadd{Geofence parameters set}
                \item \claudeadd{Failsafe protocols configured (\ac{rth}, lost link, low battery)}
                \item \claudeadd{\acp{saa} programmed as rally points}
            \end{itemize}

            \paragraph{Crew Brief} \claudeadd{Conduct final crew brief per \cref{ssec:crewbrief}.}

    \subsection{Procedures During Flight}\label{ssec:inflight}

        \subsubsection{VLOS Flight Procedures}

            \paragraph{Launch} \claudeadd{Upon completion of pre-flight and crew brief:}
            \begin{itemize}
                \item \claudeadd{\ac{rpic} announces ``Ready for launch'' and confirms \ac{lz} is clear}
                \item \claudeadd{\nameref{role:visualobserver}s confirm ready status}
                \item \claudeadd{\ac{rpic} initiates launch sequence}
                \item \claudeadd{Aircraft climbs to designated altitude and holds for system check}
                \item \claudeadd{\ac{rpic} verifies all systems nominal, then proceeds with mission}
            \end{itemize}

            \paragraph{In-Flight Monitoring} \claudeadd{During flight, the \ac{rpic} will:}
            \begin{itemize}
                \item \claudeadd{Maintain awareness of aircraft position relative to planned route}
                \item \claudeadd{Monitor telemetry for any anomalies}
                \item \claudeadd{Respond to \nameref{role:visualobserver} calls regarding aircraft position or hazards}
                \item \claudeadd{Make route adjustments as needed within approved parameters}
            \end{itemize}

            \paragraph{Contingency Actions} \claudeadd{If conditions warrant:}
            \begin{itemize}
                \item \claudeadd{For minor anomalies: continue with increased monitoring}
                \item \claudeadd{For degraded conditions: execute return to launch or nearest \ac{saa}}
                \item \claudeadd{For emergencies: follow procedures in \cref{ch:safety}}
            \end{itemize}

        \subsubsection{BVLOS Flight Procedures}

            \paragraph{Launch} \claudeadd{Upon completion of pre-flight and crew brief:}
            \begin{itemize}
                \item \claudeadd{Launch \ac{rpic} confirms receiving pilot is in position and ready}
                \item \claudeadd{\ac{rpic} announces ``Ready for launch'' and confirms \ac{lz} is clear}
                \item \claudeadd{\ac{rpic} initiates autonomous launch sequence}
                \item \claudeadd{Aircraft climbs and proceeds on programmed route}
            \end{itemize}

            \paragraph{Transit Monitoring} \claudeadd{During autonomous transit:}
            \begin{itemize}
                \item \claudeadd{\ac{rpic} monitors telemetry continuously}
                \item \claudeadd{Position reports are called at designated waypoints}
                \item \claudeadd{Receiving pilot monitors for approaching aircraft}
            \end{itemize}

            \paragraph{Approach and Landing}
            \begin{itemize}
                \item \claudeadd{Receiving pilot confirms ``Aircraft in sight'' when visual contact established}
                \item \claudeadd{\ac{rpic} monitors approach via telemetry}
                \item \claudeadd{Aircraft executes autonomous landing}
                \item \claudeadd{Receiving pilot confirms ``Touchdown'' and ``Secured''}
            \end{itemize}

            \paragraph{Abort Criteria} \claudeadd{The \ac{rpic} will abort the mission and direct the aircraft to the nearest \ac{saa} if:}
            \begin{itemize}
                \item \claudeadd{Telemetry indicates system anomaly}
                \item \claudeadd{Weather conditions deteriorate below minimums}
                \item \claudeadd{Lost link exceeds configured timeout}
                \item \claudeadd{Receiving pilot is unable to receive the aircraft}
                \item \claudeadd{Any crewmember calls ``Knock it off''}
            \end{itemize}

    \subsection{Post-Flight Procedures}\label{ssec:postflight}

        \subsubsection{} \claudeadd{Following every flight, the following procedures will be completed:}

            \paragraph{Aircraft Securing}
            \begin{itemize}
                \item \claudeadd{Power down aircraft per manufacturer procedures}
                \item \claudeadd{Remove and secure battery}
                \item \claudeadd{Conduct post-flight inspection for damage}
                \item \claudeadd{Secure payload and aircraft for transport/storage}
            \end{itemize}

            \paragraph{Notifications}
            \begin{itemize}
                \item \claudeadd{Notify 2nd AVN of flight completion (within 15 minutes)}
                \item \claudeadd{Notify \ac{wp} \ac{des} of flight completion (within 15 minutes)}
                \item \claudeadd{Report mission status to \nameref{role:projectncoic}}
            \end{itemize}

            \paragraph{Documentation}
            \begin{itemize}
                \item \claudeadd{Complete flight log entry}
                \item \claudeadd{Document any anomalies, discrepancies, or maintenance issues}
                \item \claudeadd{Download and archive telemetry data (if required)}
                \item \claudeadd{Update aircraft logbook}
            \end{itemize}

\section{Data Management}\label{sec:datamgmt}

\claudeadd{Proper documentation of flight operations supports program analysis, safety reviews, and potential expansion to other installations.}

    \subsection{Required Flight Logs}\label{ssec:flightlogs}

        \subsubsection{} \claudeadd{The following information will be recorded for every flight:}

        \begin{itemize}
            \item \claudeadd{Date and time of flight (takeoff and landing)}
            \item \claudeadd{Aircraft identification}
            \item \claudeadd{\ac{rpic} and crewmember names}
            \item \claudeadd{Mission type and route flown}
            \item \claudeadd{Payload description and weight}
            \item \claudeadd{Weather conditions}
            \item \claudeadd{Flight duration}
            \item \claudeadd{Any anomalies, deviations, or incidents}
        \end{itemize}

    \subsection{Recommended Telemetry Capture}\label{ssec:telemetrycapture}

        \subsubsection{} \claudeadd{The following telemetry data capture is recommended but not required:}

        \begin{itemize}
            \item \claudeadd{\ac{gps} track (position, altitude, speed)}
            \item \claudeadd{Battery voltage and current draw}
            \item \claudeadd{Motor performance data}
            \item \claudeadd{Link quality metrics}
        \end{itemize}

        \subsubsection{} \claudeadd{Telemetry data supports troubleshooting, route optimization, and program analysis for potential expansion.}

    \subsection{Record Retention}\label{ssec:recordretention}

        \subsubsection{} \claudeadd{Flight logs and associated documentation will be retained in accordance with Army record retention requirements. At minimum:}

        \begin{itemize}
            \item \claudeadd{Flight logs: retained for the duration of the program plus 3 years}
            \item \claudeadd{Aircraft logbooks: retained for the service life of the aircraft}
            \item \claudeadd{Incident/mishap reports: retained per DA Pamphlet 385-40}
        \end{itemize}

    \subsection{Incident Documentation}\label{ssec:incidentdocs}

        \subsubsection{} \claudeadd{Any anomaly, incident, or mishap will be documented in addition to the standard flight log. Incident documentation will include:}

        \begin{itemize}
            \item \claudeadd{Detailed narrative of the event}
            \item \claudeadd{Contributing factors identified}
            \item \claudeadd{Actions taken in response}
            \item \claudeadd{Recommendations for prevention}
        \end{itemize}

        \subsubsection{} \claudeadd{Incident reporting requirements are detailed in \cref{sec:accidentreporting}.}
