\chapter{Regulatory Framework and References}\label{ch:references}

This SOP was generated with the input of subject matter experts at \ac{kach}, \ac{usma}, \ac{srd}, \ac{usaasa}, and the \ac{faa}.  The following appendix lists the regulations, doctrine, and guidance used in the creation of this SOP along with a brief description of how each document was used.

    \section{FAA Regulations and Guidance}
        \subsubsection{Part 91}

        \subsubsection{Part 107}
        
        \subsubsection{Part 108 (Proposed)}

    \section{Joint FAA and DoD Regulations and Guidance}
        \subsubsection{Memorandum of Understanding (MOU) between DoD and FAA}
        The \textit{Memorandum of Understanding Between the Department of Defense and Federal Aviation Administration for Unmanned Aircraft System Operations in the National Airspace System} (signed January--March 2019, Annex A to HQDA EXORD 228-24) establishes the framework for DoD \ac{uas} access to the \ac{nas} outside of Restricted, Warning, or Prohibited Areas.  It supersedes the September 2013 MOA on the same subject.

        \paragraph{} Key provisions drawn upon in this SOP include:

        \begin{itemize}
            \item \textbf{Airworthiness and Operator Qualification ({\S}5.1--5.2):} All DoD UAS operating in the NAS must hold a Military Department airworthiness certification, and pilots/operators must be qualified and medically certified by the appropriate Military Department.  These requirements informed the operator certification and medical standards in this SOP.
            \item \textbf{sUAS Access to Class G Airspace Without COA ({\S}7):} DoD sUAS weighing 55\,lbs or less may operate in Class G airspace under VMC (day/night) within visual line of sight, provided a NOTAM is published 24--72 hours in advance.  This provision directly shapes the SOP's default operating conditions and NOTAM procedures.
            \item \textbf{Operations Near Non-DoD Airports ({\S}8):} UAS operations in Class B, C, D, or surface-area Class E airspace require an FAA COA and ATC authorization.  The SOP's airspace and coordination requirements reflect this constraint.
            \item \textbf{Night Operations ({\S}9):} Night flight is permitted provided the UAS meets 14 CFR 91.209 lighting requirements and crews adapt to night conditions 30 minutes prior.  The SOP's night operations procedures derive from this section.
            \item \textbf{Operations Over Populated Areas ({\S}10):} Prohibited without appropriate airworthiness certification under DoD standards, informing the SOP's restrictions on flight over personnel and populated areas.
            \item \textbf{COA Process and Timelines ({\S}6):} New COAs are processed within 60 business days; renewals within 30.  COAs are valid for 24 months.  These timelines guided the SOP's administrative planning and renewal schedules.
        \end{itemize}

        \section{DoD and Army Regulations and Guidance}
        \subsubsection{SecDef Memo --- Unleashing U.S.\ Military Drone Dominance}
        \claudeadd{\textit{Memorandum from the Secretary of Defense, ``Unleashing U.S.\ Military Drone Dominance''} (10 July 2025, signed by the Secretary of Defense) directs the rapid proliferation of Group~1 and Group~2 \ac{suas} across all \ac{dod} units.  Issued in support of Executive Order 14307 (``Unleashing American Drone Dominance,'' 6 June 2025), it rescinds the 2021 \ac{dod} guidance implementing \ac{ndaa} FY2020 {\S}848 and the 2022 Blue \ac{suas} exception-to-policy memorandum, replacing both with streamlined authorities and procurement directives.}

        \paragraph{} \claudeadd{Key provisions relevant to this SOP include:}

        \begin{itemize}
            \item \claudeadd{\textbf{Delegated Authority to O-6 Commanders:} O-6 commanders or equivalents may grant authority to operate (ATO) and may procure, test, and train with small \ac{uas} compliant with statutory limitations.  This authority may not be further delegated.  This delegation informed the SOP's approval and command authority framework.}
            \item \claudeadd{\textbf{Blue List Compliance:} All \ac{uas} must comply with the Blue List maintained by the \ac{diu} (transferring to \ac{dcma} by January 2026), which catalogues \ac{dod}-approved \ac{uas}, components, and software.  The SOP's procurement and vetting procedures align with Blue List requirements.}
            \item \claudeadd{\textbf{Cybersecurity --- Closed-Loop Networks:} All small \ac{uas} must remain in closed-loop cyber networks, cordoned from \ac{dod} networks.  This requirement reinforced the SOP's cybersecurity and data-handling provisions.}
            \item \claudeadd{\textbf{Airworthiness and Material Release Exemptions:} The Secretaries of the Military Departments shall determine airworthiness and material release requirements, exempting Group~1 and Group~2 \ac{uas} with few exceptions.  This exemption shaped the SOP's streamlined airworthiness approach for small \ac{uas}.}
            \item \claudeadd{\textbf{Training Delegation:} Group~1--2 \ac{uas} operator training and qualifications are delegated to the Military Departments with minimal additional medical standards.  \ac{dod} entities outside the Military Departments shall adhere to standards prescribed by the Secretary of the Army.  This delegation underpins the SOP's operator certification and medical standards.}
            \item \claudeadd{\textbf{Covered Foreign Entity Restrictions:} The memo reaffirms \ac{ndaa} FY2020 {\S}848, FY2023 {\S}817, and the American Security Drone Act (FY2024 {\S\S}1821--1825), permitting only limited procurement from covered foreign entities for research, training, testing, electronic warfare analysis, or counter-\ac{uas} development.  These statutory constraints informed the SOP's procurement restrictions and cybersecurity risk-mitigation procedures.}
            \item \claudeadd{\textbf{Installation Commander Responsibilities:} Installation commanders will integrate drones against their unique installation restrictions, working with the \ac{faa} to remove inappropriate range restrictions and expand spectrum approval.  This directive informed the SOP's airspace coordination and installation-specific operating area provisions.}
            \item \claudeadd{\textbf{Consumable Classification:} Group~1 and Group~2 \ac{uas} will be accounted for as consumable commodities, not durable property.  This classification influenced the SOP's property accountability and logistics procedures.}
        \end{itemize}

        \subsubsection{AR 95-1, Appendix D --- Small Unmanned Aircraft System Utilization}
        \claudeadd{\textit{Army Regulation 95--1, Flight Regulations} (22 March 2018), Appendix~D, establishes regulatory guidance for Group~1 \ac{suas} (0.55--20\,lbs) operated by personnel other than \ac{mos}-qualified unmanned aircraft operators.  Appendix~D is the primary Army regulatory authority governing \ac{suas} training, qualification, currency, airspace usage, and crew requirements.}

        \paragraph{} \claudeadd{Key provisions drawn upon in this SOP include:}

        \begin{itemize}
            \item \claudeadd{\textbf{Personnel Eligibility ({\S}D--2):} Authorized \ac{suas} operators include Regular Army, \ac{usar}, \ac{arng}, and Army civilian employees who meet qualification, training, evaluation, and currency requirements and the medical standard of AR~40--501 (without a Class~IV physical).  Government contractors require written authorization from the owning \ac{acom}/\ac{ascc}/\ac{dru}/\ac{arng}.  These eligibility criteria informed the SOP's operator qualification requirements.}
            \item \claudeadd{\textbf{Training Program and Currency ({\S\S}D--3, D--4):} The \ac{suas} \ac{atp} and currency requirements follow the appropriate \ac{mtl}.  Lapsed currency requires a \ac{pfe}; simulators may not reestablish currency.  These requirements informed the SOP's training and currency framework (subject to the D--10 exemption discussed below).}
            \item \claudeadd{\textbf{Airspace Usage ({\S}D--7):} \ac{suas} operations comply with paragraph~2--11 of AR~95--1 and applicable \ac{faa} orders.  \ac{dod} \ac{suas} weighing 0.55--20\,lbs may operate in Class~G airspace below 1,200\,ft \ac{agl} within \ac{vlos} of the operator or a certified observer, provided the \ac{uas} remains more than 5\,miles from any civil airport or heliport.  \ac{notam} publication is required no later than 24~hours in advance for non-recurring operations; standing ``blanket'' \ac{notam}s are prohibited.  These airspace provisions directly shaped the SOP's default operating conditions and \ac{notam} procedures.}
            \item \claudeadd{\textbf{Minimum Crew ({\S}D--8):} The minimum crew is one qualified operator unless the operator's manual specifies otherwise.  This requirement informed the SOP's minimum crew composition.}
            \item \claudeadd{\textbf{Certification of Operators and Master Trainers ({\S}D--9):} Master Trainers designated by the first O--6 in the chain of command certify new operators at home station.  Qualification courses are conducted at \ac{tradoc}-approved locations.  These provisions informed the SOP's operator certification pathway (subject to the D--10 exemption discussed below).}
            \item \claudeadd{\textbf{Nonstandard \ac{suas} with a Non-Tactical Mission ({\S}D--10):} This paragraph is the foundational regulatory exemption for \projectmars{}.  See discussion below.}
        \end{itemize}

        \paragraph{Paragraph D--10 Exemption} \claudeadd{Paragraph~D--10 of AR~95--1 is the single most important regulatory provision enabling \projectmars{}.  It provides that Group~1 \ac{uas} procured per Chapter~9 of AR~95--1 for non-tactical missions---explicitly including ``research and academic activities within Army research laboratories, RDECOM, military academies''---are \textbf{exempt from the qualification, evaluation, and currency requirements} of the regulation.  The owning organization retains responsibility for safe operations and compliance with applicable \ac{faa} circulars.}

        \claudeadd{For \projectmars{}, this means:}
        \begin{itemize}
            \item \claudeadd{\textbf{Exemption from AR~95--1 operator qualification, evaluation, and currency:}  \ac{usma} operators need not complete the \ac{tradoc}-approved \ac{suas} operator qualification course, semi-annual proficiency and readiness tests ({\S}D--5), or standard \ac{atp} currency requirements ({\S}D--4).  This exemption---reinforced by ALARACT~080/2024 and West Point Policy Memorandum MR-25-02 ({\S}5.c)---is the basis for the SOP's locally developed operator certification framework.}
            \item \claudeadd{\textbf{Continued \ac{faa} compliance:} The exemption does not relieve \ac{usma} of \ac{faa} obligations.  Operators must still hold an \ac{faa} Remote Pilot Certificate (14~CFR Part~107), register \ac{suas} with the \ac{faa}, and comply with all applicable \ac{faa} airspace rules and waivers.  The SOP's \ac{faa} compliance requirements derive from this continued obligation.}
            \item \claudeadd{\textbf{Safe operations responsibility:} \ac{usma} as the owning organization bears full responsibility for establishing and enforcing safe operating procedures---which is the central purpose of this SOP.}
        \end{itemize}

        \claudeadd{\Cref{fig:ar95-1-d10} reproduces Paragraph~D--10 from AR~95--1 in its entirety.}

        \begin{figure}[htbp]
            \centering
            \shadowimage[width=0.85\textwidth, page=2, trim=40 40 40 580, clip]{assets/Documents/AR 95-1 - Army Flight Regulations-Appendix D.pdf}
            \caption{\claudeadd{AR 95--1, Paragraph D--10: Nonstandard small unmanned aircraft systems with a non-tactical mission (22 March 2018).}}
            \label{fig:ar95-1-d10}
        \end{figure}

        \subsubsection{AWR}

    \section{Local Regulations and Guidance}
        \subsubsection{Academic Flight Program}
        \subsubsection{USMA sUAS SOP (SOP 385-95)}
        The \textit{Standard Operating Procedure for Safe Operations of Organic United States Military Academy Small Unmanned Aerial Systems (SUAS)} (SOP 385-95, originated 28 SEP 2017, revised 12 OCT 2023) is the primary predecessor document for this SOP.  It governs all organic \ac{usma} sUAS operations for academic purposes, covering vehicles with a takeoff weight of 30\,lbs or less. 
        
        \paragraph{}This SOP drew extensively from SOP 385-95 in the following areas:
        
        \begin{itemize}
            \item \textbf{Roles and Responsibilities ({\S}2):} The organizational structure---\ac{dafp}, Deputy Director, 2\textsuperscript{nd} AVN, Program Manager, Test Director, \ac{oic}, Flight Operator(s), and Safety Observer(s)---was adopted as the basis for the roles and responsibilities defined in this SOP.  Key constraints (e.g., cadets may not serve as OIC for outdoor operations) were carried forward.
            \item \textbf{Flight Operations Planning and Risk Management ({\S}3):} The requirement for an approved Test Plan, \ac{crm} assessment via DA Form 7566, and Safety Review Board approval informed this SOP's mission planning and risk management framework.  The proper-use review ensuring compliance with SecDef Policy Memorandum 15-002 and Section 848 (NDAA FY2020) was also incorporated.
            \item \textbf{Airworthiness ({\S}4):} The requirement for an Airworthiness Release per AR 70-62, the Bounded Operating Restrictions, and the locally delegated airworthiness determination process (Appendix J, adapted 5P Checklist) directly informed this SOP's airworthiness chapter.
            \item \textbf{Familiarization Requirements ({\S}5):} The tiered familiarization framework---annual general sUAS operations training, per-system orientation, range-specific training, and mode-specific flight control proficiency (manual/stabilized and autonomous)---was adopted for operator certification in this SOP.
            \item \textbf{Flight Operation Controls, Limitations, and Special Provisions ({\S}6):} Operational limits for NAS operations 400ft AGL, 100mph, 3sm visibility, VLOS, daylight-only with twilight lighting provisions) and failsafe protocol requirements (Loss of Link, Loss of GPS, Geofence) were carried into this SOP.  Provisions for indoor/containment operations, micro-UAS operations (\textless0.55\,lbs), multi-UAS operations, and beyond-visual-line-of-sight operations in restricted airspace were also adopted.
            \item \textbf{Emergency Response ({\S}7):} The mishap response procedures---stop mission, collect information, render aid, secure the site, contact range control and DAFP---and the distinction between reportable and non-reportable mishaps were incorporated into this SOP's safety chapter.
            \item \textbf{Local Operating Area (Appendix B):} The definition of authorized operating areas (R-5206/Range 11 and ten FAA-coordinated Class G sites on the West Point Military Installation) informed the airspace and operating area provisions of this SOP.
            \item \textbf{Cybersecurity Risk Mitigation (Appendix K):} Procedures for vetting COTS sUAS against covered-nation restrictions (NDAA FY2020 {\S}848, FY2023 {\S}817), standalone GCS device requirements, and data handling protocols were adopted into this SOP's cybersecurity and procurement provisions.
        \end{itemize}

        \subsubsection{West Point Policy Memorandum MR-25-02}
        \textit{West Point Policy Memorandum MR-25-02, Use of Small Unmanned Aircraft Systems (sUAS) on West Point Military Installation} (25 July 2025, signed by LTG Steven W.\ Gilland, Superintendent) establishes installation-wide policy for the management and use of nonofficial, non-tactical, and commercial sUAS on West Point.  It rescinds USMA Policy MA-18-35 and applies to all personnel assigned, attached, residing, visiting, or under OPCON of West Point.

        Key provisions drawn upon in this SOP include:
        \begin{itemize}
            \item \textbf{Senior Commander Approval ({\S}5.a):} All sUAS use within West Point is prohibited without prior approval from the Senior Commander or designee, on a case-by-case or recurring basis.  All sUAS must be on the approved Blue UAS Cleared Drone list per NDAA FY2020 {\S}848 and FY2023 {\S}817.  This requirement directly informed the SOP's procurement and approval processes.
            \item \textbf{FPCON Restrictions ({\S}5.b):} sUAS operations terminate immediately at FPCON Charlie or Delta and do not resume without Senior Mission Commander approval.  This constraint was incorporated into the SOP's force protection provisions.
            \item \textbf{ALARACT 080/2024 Exemption ({\S}5.c):} Academic and research activities at Military Academies are exempt from the qualification, evaluation, and currency requirements of AR 95-1, though West Point remains responsible for safe operations and FAA compliance.  This exemption is a foundational authority for the SOP's operator certification framework.
            \item \textbf{Remote Pilot Certificate and FAA Compliance ({\S}5.d--e):} Operators must hold an FAA Remote Pilot Certificate (Part 107 knowledge exam) and register sUAS with the FAA.  West Point falls under controlled airspace requiring FAA authorization.  These requirements were adopted into the SOP's operator certification and airspace chapters.
            \item \textbf{Operational Limitations ({\S}5.f):} Cloud clearance (500\,ft below, 2,000\,ft horizontal), 3\,sm minimum visibility, 400\,ft AGL maximum altitude, VLOS requirement, yield right-of-way to manned aircraft, and prohibitions on night/over-people operations without Senior Mission Commander approval and FAA waivers.  These limits directly shaped the SOP's flight operation controls.
            \item \textbf{Accident Reporting ({\S}5.g):} All sUAS accidents involving property damage or injury require immediate notification to DES Duty Desk (845-938-3333), the USMA Safety Office, and chain of command for CCIR determination.  This reporting chain was adopted into the SOP's emergency response procedures.
            \item \textbf{Flight Approval Process ({\S}7):} The tiered coordination process---sUAS Planning Group (chaired by Deputy G3), USMA sUAS Flight Request Form, weekly Helo-Ops/Flight Schedule via 2\textsuperscript{nd} AVN TASKORD, pre-/post-flight notifications to 2\textsuperscript{nd} AVN and MP Duty Desk, and RFMIS reservations for training areas west of Route 9W---was incorporated into the SOP's flight request and coordination procedures.
            \item \textbf{Restricted Areas and PAO Coordination ({\S}5.k--l):} Prohibitions on overflying formations, MEVAs, HRTs, CPRA, post housing, and access control points without Force Protection and PAO approval, plus requirements for PAO review of aerial photos/video, informed the SOP's imagery and restricted-area provisions.
            \item \textbf{Indoor/Containment Operations ({\S}7.h):} Weight limits (20\,lbs VTOL, 5\,lbs other), participant-only flight environments, containment verification, Safety Standoff Distance, and ingress/egress control requirements were carried into the SOP's indoor operations provisions.
        \end{itemize}