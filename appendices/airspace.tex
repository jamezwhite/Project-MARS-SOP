\chapter{West Point Airspace Approval}\label{ch:airspace}
This appendix outlines the approved airspace for \ac{uas} operations under \projectmars at \ac{usma}. All operations must adhere to the specified airspace restrictions and guidelines to ensure safety and regulatory compliance.

\section{Airspace Access Request}\label{sec:airspace-request}

On 21 November 2025, \ac{kach} submitted a formal request to the U.S. Army Aeronautical Services Agency (USAASA) for authorization to operate Group 1 \ac{uas} in Class G airspace within the West Point Military Reservation (WPMR). The request outlined planned operations for four approved platforms (HAR-V, Parrot ANAFI, Skydio X2D, and FLIR SkyRaider R80D) conducting intermittent flights at or below 400 feet AGL during daylight hours only. The operational area encompasses approximately 24.81 square miles within the boundaries of the WPMR. All operations were planned to be conducted under \ac{vlos} conditions with qualified operators, published \ac{notam}s, and current \ac{awr}s for each platform. The complete airspace access request is provided in \Cref{fig:vlos-request}.

\begin{figure}[p]
    \centering
    \shadowimage[width=\textwidth,height=0.85\textheight,keepaspectratio,page=1]{assets/Documents/VLOS Airspace Request.pdf}
    \caption{\ac{vlos} Airspace Access Request (Page 1 of 4)}
    \label{fig:vlos-request}
\end{figure}

\begin{figure}[p]
    \centering
    \shadowimage[width=\textwidth,height=0.85\textheight,keepaspectratio,page=2]{assets/Documents/VLOS Airspace Request.pdf}
    \caption{\ac{vlos} Airspace Access Request (Page 2 of 4)}
\end{figure}

\begin{figure}[p]
    \centering
    \shadowimage[width=\textwidth,height=0.85\textheight,keepaspectratio,page=3]{assets/Documents/VLOS Airspace Request.pdf}
    \caption{\ac{vlos} Airspace Access Request (Page 3 of 4)}
\end{figure}

\begin{figure}[p]
    \centering
    \shadowimage[width=\textwidth,height=0.85\textheight,keepaspectratio,page=4]{assets/Documents/VLOS Airspace Request.pdf}
    \caption{\ac{vlos} Airspace Access Request (Page 4 of 4)}
\end{figure}

\section{Airspace Access Approval}\label{sec:airspace-approval}

On 8 January 2026, USAASA granted authorization for \projectmars{} to operate \ac{suas} within the approved airspace under reference number 2025-ESA-00668. The approval, signed by COL Ronald C. Smith, Commanding Officer of USAASA, authorizes operations from 9 January 2026 through 8 January 2028, unless rescinded by the Commander USAASA. The authorization specifies day-only operations at a maximum altitude of 400 feet AGL and requires the unit to publish \ac{notam}s not more than 72 hours in advance but not less than 24 hours prior to operations. The unit remains responsible for maintaining current \ac{awr}s, frequency authorization, and compliance with all DoD/FAA Memorandum of Understanding requirements. The complete airspace access approval is provided in \Cref{fig:vlos-approval}.

\begin{figure}[p]
    \centering
    \shadowimage[width=\textwidth,height=0.85\textheight,keepaspectratio,page=1]{assets/Documents/VLOS Airspace Approval.pdf}
    \caption{\ac{vlos} Airspace Access Approval}
    \label{fig:vlos-approval}
\end{figure}
