\newglossaryentry{corridor}{
	name = {air corridor} ,
	description = {A restricted air route of travel specified for use by friendly aircraft and established for the purpose of preventing friendly aircraft from being fired on by friendly forces.
	\glspar\hfill\textit{Source: JP 3-52, DoD, 2014}}}

\newglossaryentry{aircraft}{
	name = {aircraft} ,
	description = {A device that is used, or intended to be used, for flight.
	\glspar\hfill\textit{Source: Pilot's Handbook of Aeronautical Knowledge, FAA, November 2023}}}

\newglossaryentry{airframe}{
	name = {airframe} ,
	description = {The fuselage, booms, nacelles, cowlings, fairings, airfoil surfaces (including rotors but excluding propellers and rotating airfoils of engines), and landing gear of an aircraft and their accessories and controls.
	\glspar\hfill\textit{Source: 14 CFR Subsection 1.1, US Congress, September 2025}}}

\newglossaryentry{automated}{
	name = {automated} ,
	description = {Operated by machines or computers without direct human control.
	\glspar\hfill\textit{Source: Dictionary, Merriam-Webster, 2024}}}

\newglossaryentry{adsb}{
	name = {Automatic Dependent Surveillance - Broadcast (ADS-B)} ,
	description = {A function of an aircraft or vehicle that preiodically broadcasts its state vector (i.e., horizontal and vertical position, horizontal and vertical velocity) and other information.
	\glspar\hfill\textit{Source: Pilot's Handbook of Aeronautical Knowledge, FAA, November 2023}}}

\newglossaryentry{autonomousflight}{
	name = {autonomous flight} ,
	description = {The operation of a /ac{uas} where a complex function makes decisions and performs actions without direct human input, enabled by a run-time assurance (RTA) architecture that safely bounds the flight behavior.
	\glspar\hfill\textit{Source: ASTM F3269-17, ASTM International, 2017}}}

\newglossaryentry{bvlos}{
	name = {Beyond Visual Line of Sight (BVLOS)} ,
	description = {The operation of a UAS beyond the visual capability of the flight crew members (i.e., remote pilot in command [RPIC], the person manipulating the controls, and visual observer [VO]), if used to see the aircraft with vision unaided by any device other than corrective lenses, spectacles, and contact lenses.
	\glspar\hfill\textit{Source: Pilot/Controller Glossary, FAA, August 2025}}}

\newglossaryentry{coa}{
	name = {Certificate or Waiver of Authorization (CoA)} ,
	description = {An FAA grant of approval for a specific flight operation or airspace authorization or waiver.
	\glspar\hfill\textit{Source: Pilot/Controller Glossary, FAA, August 2025}}}

\newglossaryentry{classg}{
	name = {Class G airspace} ,
	description = {Airspace that is uncontrolled, except when associated with a temporary control tower, and has not been designated as Class A, Class B, Class C, Class D, or Class E airspace.
	\glspar\hfill\textit{Source: Pilot's Handbook of Aeronautical Knowledge, FAA, November 2023}}}

\newglossaryentry{controlled}{
	name = {controlled airspace} ,
	description = {An airspace of defined dimensions within which ATC service is provided to IFR and VFR flights in accordance with the airspace classification. It includes Class A, Class B, Class C, Class D, and Class E airspace.
	\glspar\hfill\textit{Source: Pilot's Handbook of Aeronautical Knowledge, FAA, November 2023}}}

\newglossaryentry{crewmember}{
	name = {crewmember (UAS)} ,
	description = {A person assigned to perform an operational duty. A UAS crewmember includes the remote pilot in command, the person manipulating the controls, and visual observers but may also include other persons as appropriate or required to ensure the safe operation of the UAS (e.g., sensor operator, ground control station operator).
	\glspar\hfill\textit{Source: Pilot/Controller Glossary, FAA, August 2025}}}

\newglossaryentry{mou}{
	name = {DoD / FAA Memorandum of Understanding (MoU)} ,
	description = {a /ac{mou} between the /ac{dod} and /ac{faa} that sets forth provisions that allow for increased /ac{dod} /ac{uas} use in the /ac{nas} outside of restricted, warning, or prohibited areas. This /ac{mou} defines many collaboration methods, and lists specific requirements that the /ac{dod} and /ac{faa} must meet.
	\glspar\hfill\textit{Source: /ac{mou} between the /ac{dod} and /ac{faa} for /{uas} in the /ac{nas}, Jointly published, May 2019}}}

\newglossaryentry{drone}{
	name = {drone} ,
	description = {See \term{uas}.
	\glspar\hfill\textit{Source: }}}

\newglossaryentry{flightcorridor}{
	name = {flight corridor} ,
	description = {A defined three-dimensional airspace, of defined width, which has been set aside for flights along a particular route.
	\glspar\hfill\textit{Source: ICAO Document 4444, International Civil Aviation Organization, 2016}}}

\newglossaryentry{flightpath}{
	name = {flight path} ,
	description = {The line, course, or track along which an aircraft is flying or is intended to be flown.
	\glspar\hfill\textit{Source: Pilot's Handbook of Aeronautical Knowledge, FAA, November 2023}}}

\newglossaryentry{geofence}{
	name = {geo-fence} ,
	description = {A virtual perimeter for a real-world geographic area that can be dynamically generated or correspond to a pre-set geographic boundary.
	\glspar\hfill\textit{Source: ASTM F3002-14a, ASTM International, 2014}}}

\newglossaryentry{groups}{
	name = {Group (\ac{dod} \ac{uas} Classification)} ,
	description = {The \ac{uas} group system establishes the foundation for joint \ac{uas} terminology. It provides a common reference to compare \ac{uas}. \ac{uas} are grouped based on the physical and performance characteristics of weight, operating altitude, and airspeed. \ac{uas} groups are determined without regard for payload, mission, command relationship, or Service. All \ac{uas} fall into one of five groups.
	\glspar\hfill\textit{Source: AR 95-1, DA, March 2018}}}

\newglossaryentry{group1}{
	name = {Group 1 \ac{uas}} ,
	description = {Typically hand-launched, self contained, portable systemsemployed for a small unit or base security. They are capable of providing “overthe hill” or “around the corner” reconnaissance and surveillance. They operatewithin visual range and are analogous to radio-controlled model airplanes ascovered in AC 91-57.30. Typically weigh less than 20lbs \ac{mgtow}.
	\glspar\hfill\textit{Source:  DoD UAS Airspace Integration Plan, 2004}}}

\newglossaryentry{group2}{
	name = {Group 2 \ac{uas}} ,
	description = {Small to medium in size and usually support brigade and intelligence, surveillance, reconnaissance, and target acquisition requirements. They usually operate from unimproved areas and launched via catapult. Payloads may include a sensor ball with electro-optic / infrared(EO/IR) and laser range finder/designator (LRF/D) capability. They typically perform special purpose operations or routine operations within a specific set of restrictions. Typically weigh between 21lbs and 55lbs \ac{mgtow}.
	\glspar\hfill\textit{Source:  DoD UAS Airspace Integration Plan, 2004}}}

\newglossaryentry{group3}{
	name = {Group 3 \ac{uas}} ,
	description = {Operate at medium altitudes with medium to long range and endurance. Their payloads may include a sensor ball with EO/IR, LRF/D,signal intelligence (SIGINT), communications relay, and chemical biological radiological nuclear explosive (CBRNE) detection. They usually operate from unimproved areas and may not require an improved runway. Typically weigh more than 55lbs, but less than 1320lbs \ac{mgtow}.
	\glspar\hfill\textit{Source:  DoD UAS Airspace Integration Plan, 2004}}}

\newglossaryentry{group4}{
	name = {Group 4 \ac{uas}} ,
	description = {Relatively large UAS that operate at medium to high altitudes and have extended range and endurance. They normally require improved areas for launch and recovery, beyond line-of-sight (BLOS) communications, andhave stringent airspace operations requirements. Payloads may include EO/IR sensors, radars, lasers, communications relay, SIGINT, Automatic Identification System (AIS), and weapons. Typically weigh more than 1320lbs \ac{mgtow}.
	\glspar\hfill\textit{Source:  DoD UAS Airspace Integration Plan, 2004}}}

\newglossaryentry{group5}{
	name = {Group 5 \ac{uas}} ,
	description = {Include the largest systems, operate at medium to high altitudes, and have the greatest range, endurance, and airspeed capabilities. They require improved areas for launch and recovery, BLOS communications, andthe most stringent airspace operations requirements. Group 5 UAS perform specialized missions such as broad area surveillance and penetrating attacks. Typically weigh more than 1320lbs \ac{mgtow}.
	\glspar\hfill\textit{Source:  DoD UAS Airspace Integration Plan, 2004}}}

\newglossaryentry{manipulator}{
	name = {manipulator of the controls} ,
	description = {A person who operates the controls of an aircraft during flight time.
	\glspar\hfill\textit{Source:  AC 61-65H, Federal Aviation Administration, September 2018}}}

\newglossaryentry{mgtow}{
	name = {Maximum Gross Takeoff Weight (MGTOW)} ,
	description = {The maximum allowable weight for takeoff, inclusive of airframe, fuel, and cargo.
	\glspar\hfill\textit{Source: Pilot's Handbook of Aeronautical Knowledge, FAA, November 2023}}}

\newglossaryentry{msl}{
	name = {mean sea level} ,
	description = {The average height of the surface of the sea at a particular location for all stages of the tide over a 19-year period.
	\glspar\hfill\textit{Source: Pilot's Handbook of Aeronautical Knowledge, FAA, November 2023}}}

\newglossaryentry{nas}{
	name = {National Airspace System (NAS)} ,
	description = {The common network of United States airspace—air navigation facilities, equipment and services, airports or landing areas; aeronautical charts, information and services; rules, regulations and procedures, technical information; and manpower and material.
	\glspar\hfill\textit{Source: Pilot's Handbook of Aeronautical Knowledge, FAA, November 2023}}}

\newglossaryentry{nsufr}{
	name = {National Security Unmanned Aircraft System Flight Restriction (NSUFR)} ,
	description = {Temporary flight restrictions that prohibit UAS operations within specified distances of certain security sensitive facilities to address national security concerns.
	\glspar\hfill\textit{Source: FAA UAS Data Exchange (LAANC), FAA, 2024}}}

\newglossaryentry{notam}{
	name = {Notice to Airmen (NOTAM)} ,
	description = {A notice filed with an aviation authority to alert aircraft pilots of any hazards en route or at a specific location. The authority in turn provides means of disseminating relevant NOTAMs to pilots.
	\glspar\hfill\textit{Source: Pilot's Handbook of Aeronautical Knowledge, FAA, November 2023}}}

\newglossaryentry{oop}{
	name = {Operations Over People (OOP)} ,
	description = {Operations of small unmanned aircraft over people.
	\glspar\hfill\textit{Source: Pilot/Controller Glossary, FAA, August 2025}}}

\newglossaryentry{operator}{
	name = {operator (UAS)} ,
	description = {The owner and/or remote pilot of a UAS.
	\glspar\hfill\textit{Source: Pilot/Controller Glossary, FAA, August 2025}}}

\newglossaryentry{part107}{
	name = {Part 107} ,
	description = {The Federal Aviation Regulation that establishes requirements for the operation of small unmanned aircraft systems (sUAS) in the National Airspace System.
	\glspar\hfill\textit{Source: 14 CFR Part 107, US Congress, August 2016}}}

\newglossaryentry{part91}{
	name = {Part 91} ,
	description = {General operating and flight rules that apply to aircraft operations within the United States and its territories.
	\glspar\hfill\textit{Source: 14 CFR Part 91, US Congress, April 2024}}}

\newglossaryentry{pilot}{
	name = {pilot} ,
	description = {The term “pilot” means an individual who has final authority and responsibility for the operation and safety of the flight or any other flight deck crew member.
	\glspar\hfill\textit{Source: 14 CFR Subsection 44921, US Congress, October 2018}}}

\newglossaryentry{pic}{
	name = {Pilot in Command (PIC)} ,
	description = {The pilot responsible for the operation and safety of an aircraft during flight time.
	\glspar\hfill\textit{Source: Pilot/Controller Glossary, FAA, August 2025}}}

\newglossaryentry{piloted}{
	name = {piloted} ,
	description = {Flight of an unmanned aircraft that allows pilot intervention in the management of the flight path.
	\glspar\hfill\textit{Source: ASTM F3269-17, ASTM International, 2017}}}

\newglossaryentry{rp}{
	name = {remote pilot} ,
	description = {Pilot of a UAS who is not operating as a recreational flyer under 49 USC §44809, the Exception for Limited Recreational Operations of Unmanned Aircraft.
	\glspar\hfill\textit{Source: Pilot/Controller Glossary, FAA, August 2025}}}

\newglossaryentry{rpic}{
	name = {Remote Pilot In Command (RPIC)} ,
	description = {The RPIC is directly responsible for and is the final authority as to the operation of the unmanned aircraft system.
	\glspar\hfill\textit{Source: Pilot/Controller Glossary, FAA, August 2025}}}

\newglossaryentry{tfr}{
	name = {Temporary Flight Restriction (TFR)} ,
	description = {Restriction to flight imposed in order to: 1. Protect persons and property in the air or on the surface from an existing or imminent flight associated hazard; 2. Provide a safe environment for the operation of disaster relief aircraft; 3. Prevent an unsafe congestion of sightseeing aircraft above an incident; 4. Protect the President, Vice President, or other public figures; and, 5. Provide a safe environment for space agency operations. Pilots are expected to check appropriate NOTAMs during flight planning when conducting flight in an area where a temporary flight restriction is in effect.
	\glspar\hfill\textit{Source: Pilot's Handbook of Aeronautical Knowledge, FAA, November 2023}}}

\newglossaryentry{uncontrolled}{
	name = {uncontrolled airspace} ,
	description = {Class G airspace that has not been designated as Class A, B, C, D, or E. It is airspace in which air traffic control has no authority or responsibility to control air traffic; however, pilots should remember there are VFR minimums which apply to this airspace.
	\glspar\hfill\textit{Source: Pilot's Handbook of Aeronautical Knowledge, FAA, November 2023}}}

\newglossaryentry{uav}{
	name = {Unmanned Aircraft (UA) / Unmanned Aerial Vehicle (UAV)} ,
	description = {A device used or intended to be used for flight that has no onboard pilot. This device can be any type of airplane, helicopter, airship, or powered-lift aircraft. Unmanned free balloons, moored balloons, tethered aircraft, gliders, and unmanned rockets are not considered to be a UA.
	\glspar\hfill\textit{Source: Pilot/Controller Glossary, FAA, August 2025}}}

\newglossaryentry{uas}{
	name = {Unmanned Aircraft System (UAS)} ,
	description = {An unmanned aircraft and its associated elements related to safe operations, which may include control stations (ground, ship, or air based), control links, support equipment, payloads, flight termination systems, and launch/recovery equipment. It consists of three elements: unmanned aircraft, control station, and data link
	\glspar\hfill\textit{Source: Pilot/Controller Glossary, FAA, August 2025}}}

\newglossaryentry{vlos}{
	name = {Visual Line of Sight (VLOS)} ,
	description = {Condition of operations wherein the operator maintains continuous, unaided visual contact with the unmanned aircraft
	\glspar\hfill\textit{Source: Pilot/Controller Glossary, FAA, August 2025}}}

\newglossaryentry{vo}{
	name = {visual observer} ,
	description = {A person who is designated by the remote pilot in command to assist the remote pilot in command and the person operating the flight controls of the small UAS (sUAS) to see and avoid other air traffic or objects aloft or on the ground.
	\glspar\hfill\textit{Source: Pilot/Controller Glossary, FAA, August 2025}}}

